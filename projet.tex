% Options for packages loaded elsewhere
\PassOptionsToPackage{unicode}{hyperref}
\PassOptionsToPackage{hyphens}{url}
%
\documentclass[
]{article}
\usepackage{amsmath,amssymb}
\usepackage{lmodern}
\usepackage{iftex}
\ifPDFTeX
  \usepackage[T1]{fontenc}
  \usepackage[utf8]{inputenc}
  \usepackage{textcomp} % provide euro and other symbols
\else % if luatex or xetex
  \usepackage{unicode-math}
  \defaultfontfeatures{Scale=MatchLowercase}
  \defaultfontfeatures[\rmfamily]{Ligatures=TeX,Scale=1}
\fi
% Use upquote if available, for straight quotes in verbatim environments
\IfFileExists{upquote.sty}{\usepackage{upquote}}{}
\IfFileExists{microtype.sty}{% use microtype if available
  \usepackage[]{microtype}
  \UseMicrotypeSet[protrusion]{basicmath} % disable protrusion for tt fonts
}{}
\makeatletter
\@ifundefined{KOMAClassName}{% if non-KOMA class
  \IfFileExists{parskip.sty}{%
    \usepackage{parskip}
  }{% else
    \setlength{\parindent}{0pt}
    \setlength{\parskip}{6pt plus 2pt minus 1pt}}
}{% if KOMA class
  \KOMAoptions{parskip=half}}
\makeatother
\usepackage{xcolor}
\usepackage[margin=1in]{geometry}
\usepackage{color}
\usepackage{fancyvrb}
\newcommand{\VerbBar}{|}
\newcommand{\VERB}{\Verb[commandchars=\\\{\}]}
\DefineVerbatimEnvironment{Highlighting}{Verbatim}{commandchars=\\\{\}}
% Add ',fontsize=\small' for more characters per line
\usepackage{framed}
\definecolor{shadecolor}{RGB}{248,248,248}
\newenvironment{Shaded}{\begin{snugshade}}{\end{snugshade}}
\newcommand{\AlertTok}[1]{\textcolor[rgb]{0.94,0.16,0.16}{#1}}
\newcommand{\AnnotationTok}[1]{\textcolor[rgb]{0.56,0.35,0.01}{\textbf{\textit{#1}}}}
\newcommand{\AttributeTok}[1]{\textcolor[rgb]{0.77,0.63,0.00}{#1}}
\newcommand{\BaseNTok}[1]{\textcolor[rgb]{0.00,0.00,0.81}{#1}}
\newcommand{\BuiltInTok}[1]{#1}
\newcommand{\CharTok}[1]{\textcolor[rgb]{0.31,0.60,0.02}{#1}}
\newcommand{\CommentTok}[1]{\textcolor[rgb]{0.56,0.35,0.01}{\textit{#1}}}
\newcommand{\CommentVarTok}[1]{\textcolor[rgb]{0.56,0.35,0.01}{\textbf{\textit{#1}}}}
\newcommand{\ConstantTok}[1]{\textcolor[rgb]{0.00,0.00,0.00}{#1}}
\newcommand{\ControlFlowTok}[1]{\textcolor[rgb]{0.13,0.29,0.53}{\textbf{#1}}}
\newcommand{\DataTypeTok}[1]{\textcolor[rgb]{0.13,0.29,0.53}{#1}}
\newcommand{\DecValTok}[1]{\textcolor[rgb]{0.00,0.00,0.81}{#1}}
\newcommand{\DocumentationTok}[1]{\textcolor[rgb]{0.56,0.35,0.01}{\textbf{\textit{#1}}}}
\newcommand{\ErrorTok}[1]{\textcolor[rgb]{0.64,0.00,0.00}{\textbf{#1}}}
\newcommand{\ExtensionTok}[1]{#1}
\newcommand{\FloatTok}[1]{\textcolor[rgb]{0.00,0.00,0.81}{#1}}
\newcommand{\FunctionTok}[1]{\textcolor[rgb]{0.00,0.00,0.00}{#1}}
\newcommand{\ImportTok}[1]{#1}
\newcommand{\InformationTok}[1]{\textcolor[rgb]{0.56,0.35,0.01}{\textbf{\textit{#1}}}}
\newcommand{\KeywordTok}[1]{\textcolor[rgb]{0.13,0.29,0.53}{\textbf{#1}}}
\newcommand{\NormalTok}[1]{#1}
\newcommand{\OperatorTok}[1]{\textcolor[rgb]{0.81,0.36,0.00}{\textbf{#1}}}
\newcommand{\OtherTok}[1]{\textcolor[rgb]{0.56,0.35,0.01}{#1}}
\newcommand{\PreprocessorTok}[1]{\textcolor[rgb]{0.56,0.35,0.01}{\textit{#1}}}
\newcommand{\RegionMarkerTok}[1]{#1}
\newcommand{\SpecialCharTok}[1]{\textcolor[rgb]{0.00,0.00,0.00}{#1}}
\newcommand{\SpecialStringTok}[1]{\textcolor[rgb]{0.31,0.60,0.02}{#1}}
\newcommand{\StringTok}[1]{\textcolor[rgb]{0.31,0.60,0.02}{#1}}
\newcommand{\VariableTok}[1]{\textcolor[rgb]{0.00,0.00,0.00}{#1}}
\newcommand{\VerbatimStringTok}[1]{\textcolor[rgb]{0.31,0.60,0.02}{#1}}
\newcommand{\WarningTok}[1]{\textcolor[rgb]{0.56,0.35,0.01}{\textbf{\textit{#1}}}}
\usepackage{graphicx}
\makeatletter
\def\maxwidth{\ifdim\Gin@nat@width>\linewidth\linewidth\else\Gin@nat@width\fi}
\def\maxheight{\ifdim\Gin@nat@height>\textheight\textheight\else\Gin@nat@height\fi}
\makeatother
% Scale images if necessary, so that they will not overflow the page
% margins by default, and it is still possible to overwrite the defaults
% using explicit options in \includegraphics[width, height, ...]{}
\setkeys{Gin}{width=\maxwidth,height=\maxheight,keepaspectratio}
% Set default figure placement to htbp
\makeatletter
\def\fps@figure{htbp}
\makeatother
\setlength{\emergencystretch}{3em} % prevent overfull lines
\providecommand{\tightlist}{%
  \setlength{\itemsep}{0pt}\setlength{\parskip}{0pt}}
\setcounter{secnumdepth}{-\maxdimen} % remove section numbering
\ifLuaTeX
  \usepackage{selnolig}  % disable illegal ligatures
\fi
\IfFileExists{bookmark.sty}{\usepackage{bookmark}}{\usepackage{hyperref}}
\IfFileExists{xurl.sty}{\usepackage{xurl}}{} % add URL line breaks if available
\urlstyle{same} % disable monospaced font for URLs
\hypersetup{
  pdftitle={Projet Analyse de données},
  pdfauthor={Rudio et Léo-Paul},
  hidelinks,
  pdfcreator={LaTeX via pandoc}}

\title{Projet Analyse de données}
\author{Rudio et Léo-Paul}
\date{2023-05-09}

\begin{document}
\maketitle

\hypertarget{pruxe9sentation-du-projet-et-du-jeu-de-donnuxe9es}{%
\section{Présentation du projet et du jeu de
données}\label{pruxe9sentation-du-projet-et-du-jeu-de-donnuxe9es}}

Le jeu de données est constitués d'informations sur la vie d'étudiants
dans une université du Portugal. Ces informations vont de leur résultats
universitaires, leur vie familiale à leur consommation d'alcool. Le jeu
a été construit à partir d'une enquête menée auprès d'étudiant en
mathématiques et en portugais.

L'objectif serait alors d'analyser le jeu de données afin de comprendre
les facteurs qui impactent la réussite scolaire de ces étudiants.
L'intérêt du jeu est la grande variété de facteurs proposée qui permet
de courvrir un maximum d'hypothèsesn, notamment celle sur la
consommation d'alcool proposée directement par le nom du jeu de données.

Voici les variabales présentent dans ce jeu de données ;

\begin{itemize}
\tightlist
\item
  \textbf{school} - student's school (binary: `GP' - Gabriel Pereira or
  `MS' - Mousinho da Silveira)
\item
  \textbf{sex} - student's sex (binary: `F' - female or `M' - male)
\item
  \textbf{age} - student's age (numeric: from 15 to 22)
\item
  \textbf{address} - student's home address type (binary: `U' - urban or
  `R' - rural)
\item
  \textbf{famsize} - family size (binary: `LE3' - less or equal to 3 or
  `GT3' - greater than 3)
\item
  \textbf{Pstatus} - parent's cohabitation status (binary: `T' - living
  together or `A' - apart)
\item
  \textbf{Medu} - mother's education (numeric: 0 - none, 1 - primary
  education (4th grade), 2 -- 5th to 9th grade, 3 -- secondary education
  or 4 -- higher education)
\item
  \textbf{Fedu} - father's education (numeric: 0 - none, 1 - primary
  education (4th grade), 2 -- 5th to 9th grade, 3 -- secondary education
  or 4 -- higher education)
\item
  \textbf{Mjob} - mother's job (nominal: `teacher', `health' care
  related, civil `services' (e.g.~administrative or police), `at\_home'
  or `other')
\item
  \textbf{Fjob} - father's job (nominal: `teacher', `health' care
  related, civil `services' (e.g.~administrative or police), `at\_home'
  or `other')
\item
  \textbf{reason} - reason to choose this school (nominal: close to
  `home', school `reputation', `course' preference or `other')
\item
  \textbf{guardian} - student's guardian (nominal: `mother', `father' or
  `other')
\item
  \textbf{traveltime} - home to school travel time (numeric: 1 -
  \textless15 min., 2 - 15 to 30 min., 3 - 30 min. to 1 hour, or 4 -
  \textgreater1 hour)
\item
  \textbf{studytime} - weekly study time (numeric: 1 - \textless2 hours,
  2 - 2 to 5 hours, 3 - 5 to 10 hours, or 4 - \textgreater10 hours)
\item
  \textbf{failures} - number of past class failures (numeric: n if
  1\textless=n\textless3, else 4)
\item
  \textbf{schoolsup} - extra educational support (binary: yes or no)
\item
  \textbf{famsup} - family educational support (binary: yes or no)
\item
  \textbf{paid} - extra paid classes within the course subject (Math or
  Portuguese) (binary: yes or no)
\item
  \textbf{activities} - extra-curricular activities (binary: yes or no)
\item
  \textbf{nursery} - attended nursery school (binary: yes or no)
\item
  \textbf{higher} - wants to take higher education (binary: yes or no)
\item
  \textbf{internet} - Internet access at home (binary: yes or no)
\item
  \textbf{romantic} - with a romantic relationship (binary: yes or no)
\item
  \textbf{famrel} - quality of family relationships (numeric: from 1 -
  very bad to 5 - excellent)
\item
  \textbf{freetime} - free time after school (numeric: from 1 - very low
  to 5 - very high)
\item
  \textbf{goout} - going out with friends (numeric: from 1 - very low to
  5 - very high)
\item
  \textbf{Dalc} - workday alcohol consumption (numeric: from 1 - very
  low to 5 - very high)
\item
  \textbf{Walc} - weekend alcohol consumption (numeric: from 1 - very
  low to 5 - very high)
\item
  \textbf{health} - current health status (numeric: from 1 - very bad to
  5 - very good)
\item
  \textbf{absences} - number of school absences (numeric: from 0 to 93)
\end{itemize}

These grades are related with the course subject, Math or Portuguese: -
\textbf{G1} - first period grade (numeric: from 0 to 20) - \textbf{G2} -
second period grade (numeric: from 0 to 20) - \textbf{G3} - final grade
(numeric: from 0 to 20, output target)

Au cours de ce projet, nous nous concentrons sur la variable G3 qui est
la variable de sortie représentant la note finale des élèves. Il
s'agirait donc d'un problème de régression sur la variables G3 ou même
plus généralement un problème de classification.

Voici les étapes que nous allons suivre :

\begin{enumerate}
\def\labelenumi{\arabic{enumi}.}
\tightlist
\item
  Identifier les variables significatives
\item
  Appliquer des méthodes de classification sur la réussite scolaire
\item
  Effectuer une regression linéaires pour prédire G3
\item
  Comparer des méthodes de machine learning pour prédire G3
\end{enumerate}

\hypertarget{chargement-des-donnuxe9es}{%
\subsection{1.Chargement des données}\label{chargement-des-donnuxe9es}}

\begin{Shaded}
\begin{Highlighting}[]
\CommentTok{\# Chargement de la base de données}
\NormalTok{df.mat}\OtherTok{=}\FunctionTok{read.table}\NormalTok{(}\StringTok{"student{-}mat.csv"}\NormalTok{,}\AttributeTok{sep=}\StringTok{","}\NormalTok{,}\AttributeTok{header=}\ConstantTok{TRUE}\NormalTok{,}\AttributeTok{as.is =} \ConstantTok{FALSE}\NormalTok{)}
\NormalTok{df.por}\OtherTok{=}\FunctionTok{read.table}\NormalTok{(}\StringTok{"student{-}por.csv"}\NormalTok{,}\AttributeTok{sep=}\StringTok{","}\NormalTok{,}\AttributeTok{header=}\ConstantTok{TRUE}\NormalTok{,}\AttributeTok{as.is =} \ConstantTok{FALSE}\NormalTok{)}

\CommentTok{\# Etudiants qui appartiennent aux deux cours}
\NormalTok{both}\OtherTok{=} \FunctionTok{merge}\NormalTok{(df.mat,df.por,}\AttributeTok{by=}\FunctionTok{c}\NormalTok{(}\StringTok{"school"}\NormalTok{,}\StringTok{"sex"}\NormalTok{,}\StringTok{"age"}\NormalTok{,}\StringTok{"address"}\NormalTok{,}\StringTok{"famsize"}\NormalTok{,}\StringTok{"Pstatus"}\NormalTok{,}\StringTok{"Medu"}\NormalTok{,}\StringTok{"Fedu"}\NormalTok{,}\StringTok{"Mjob"}\NormalTok{,}\StringTok{"Fjob"}\NormalTok{,}\StringTok{"reason"}\NormalTok{,}\StringTok{"nursery"}\NormalTok{,}\StringTok{"internet"}\NormalTok{))}

\CommentTok{\# Concaténation des deux dataframes}
\NormalTok{df }\OtherTok{=} \FunctionTok{rbind}\NormalTok{(df.mat,df.por)}
\FunctionTok{head}\NormalTok{(df)}
\end{Highlighting}
\end{Shaded}

\begin{verbatim}
##   school sex age address famsize Pstatus Medu Fedu     Mjob     Fjob     reason
## 1     GP   F  18       U     GT3       A    4    4  at_home  teacher     course
## 2     GP   F  17       U     GT3       T    1    1  at_home    other     course
## 3     GP   F  15       U     LE3       T    1    1  at_home    other      other
## 4     GP   F  15       U     GT3       T    4    2   health services       home
## 5     GP   F  16       U     GT3       T    3    3    other    other       home
## 6     GP   M  16       U     LE3       T    4    3 services    other reputation
##   guardian traveltime studytime failures schoolsup famsup paid activities
## 1   mother          2         2        0       yes     no   no         no
## 2   father          1         2        0        no    yes   no         no
## 3   mother          1         2        3       yes     no  yes         no
## 4   mother          1         3        0        no    yes  yes        yes
## 5   father          1         2        0        no    yes  yes         no
## 6   mother          1         2        0        no    yes  yes        yes
##   nursery higher internet romantic famrel freetime goout Dalc Walc health
## 1     yes    yes       no       no      4        3     4    1    1      3
## 2      no    yes      yes       no      5        3     3    1    1      3
## 3     yes    yes      yes       no      4        3     2    2    3      3
## 4     yes    yes      yes      yes      3        2     2    1    1      5
## 5     yes    yes       no       no      4        3     2    1    2      5
## 6     yes    yes      yes       no      5        4     2    1    2      5
##   absences G1 G2 G3
## 1        6  5  6  6
## 2        4  5  5  6
## 3       10  7  8 10
## 4        2 15 14 15
## 5        4  6 10 10
## 6       10 15 15 15
\end{verbatim}

\hypertarget{nettoyage-et-vuxe9rification-des-donnuxe9es}{%
\subsection{2. Nettoyage et vérification des
données}\label{nettoyage-et-vuxe9rification-des-donnuxe9es}}

Le jeu est composé de 33 variables dont 17 qualitatives et 16
quantitatives. On calcule la moyenne pour chaque élève, et on rajoute
une variable pour la réussite scolaire.

\begin{Shaded}
\begin{Highlighting}[]
\FunctionTok{print}\NormalTok{(}\FunctionTok{str}\NormalTok{(df))}
\end{Highlighting}
\end{Shaded}

\begin{verbatim}
## 'data.frame':    1044 obs. of  33 variables:
##  $ school    : Factor w/ 2 levels "GP","MS": 1 1 1 1 1 1 1 1 1 1 ...
##  $ sex       : Factor w/ 2 levels "F","M": 1 1 1 1 1 2 2 1 2 2 ...
##  $ age       : int  18 17 15 15 16 16 16 17 15 15 ...
##  $ address   : Factor w/ 2 levels "R","U": 2 2 2 2 2 2 2 2 2 2 ...
##  $ famsize   : Factor w/ 2 levels "GT3","LE3": 1 1 2 1 1 2 2 1 2 1 ...
##  $ Pstatus   : Factor w/ 2 levels "A","T": 1 2 2 2 2 2 2 1 1 2 ...
##  $ Medu      : int  4 1 1 4 3 4 2 4 3 3 ...
##  $ Fedu      : int  4 1 1 2 3 3 2 4 2 4 ...
##  $ Mjob      : Factor w/ 5 levels "at_home","health",..: 1 1 1 2 3 4 3 3 4 3 ...
##  $ Fjob      : Factor w/ 5 levels "at_home","health",..: 5 3 3 4 3 3 3 5 3 3 ...
##  $ reason    : Factor w/ 4 levels "course","home",..: 1 1 3 2 2 4 2 2 2 2 ...
##  $ guardian  : Factor w/ 3 levels "father","mother",..: 2 1 2 2 1 2 2 2 2 2 ...
##  $ traveltime: int  2 1 1 1 1 1 1 2 1 1 ...
##  $ studytime : int  2 2 2 3 2 2 2 2 2 2 ...
##  $ failures  : int  0 0 3 0 0 0 0 0 0 0 ...
##  $ schoolsup : Factor w/ 2 levels "no","yes": 2 1 2 1 1 1 1 2 1 1 ...
##  $ famsup    : Factor w/ 2 levels "no","yes": 1 2 1 2 2 2 1 2 2 2 ...
##  $ paid      : Factor w/ 2 levels "no","yes": 1 1 2 2 2 2 1 1 2 2 ...
##  $ activities: Factor w/ 2 levels "no","yes": 1 1 1 2 1 2 1 1 1 2 ...
##  $ nursery   : Factor w/ 2 levels "no","yes": 2 1 2 2 2 2 2 2 2 2 ...
##  $ higher    : Factor w/ 2 levels "no","yes": 2 2 2 2 2 2 2 2 2 2 ...
##  $ internet  : Factor w/ 2 levels "no","yes": 1 2 2 2 1 2 2 1 2 2 ...
##  $ romantic  : Factor w/ 2 levels "no","yes": 1 1 1 2 1 1 1 1 1 1 ...
##  $ famrel    : int  4 5 4 3 4 5 4 4 4 5 ...
##  $ freetime  : int  3 3 3 2 3 4 4 1 2 5 ...
##  $ goout     : int  4 3 2 2 2 2 4 4 2 1 ...
##  $ Dalc      : int  1 1 2 1 1 1 1 1 1 1 ...
##  $ Walc      : int  1 1 3 1 2 2 1 1 1 1 ...
##  $ health    : int  3 3 3 5 5 5 3 1 1 5 ...
##  $ absences  : int  6 4 10 2 4 10 0 6 0 0 ...
##  $ G1        : int  5 5 7 15 6 15 12 6 16 14 ...
##  $ G2        : int  6 5 8 14 10 15 12 5 18 15 ...
##  $ G3        : int  6 6 10 15 10 15 11 6 19 15 ...
## NULL
\end{verbatim}

\begin{Shaded}
\begin{Highlighting}[]
\FunctionTok{print}\NormalTok{(}\FunctionTok{nrow}\NormalTok{(df))}
\end{Highlighting}
\end{Shaded}

\begin{verbatim}
## [1] 1044
\end{verbatim}

\begin{Shaded}
\begin{Highlighting}[]
\CommentTok{\# factor}
\CommentTok{\# df$famrel=factor(df$famrel)}
\NormalTok{df}\SpecialCharTok{$}\NormalTok{Dalc}\OtherTok{=}\FunctionTok{factor}\NormalTok{(df}\SpecialCharTok{$}\NormalTok{Dalc)}
\NormalTok{df}\SpecialCharTok{$}\NormalTok{Walc}\OtherTok{=}\FunctionTok{factor}\NormalTok{(df}\SpecialCharTok{$}\NormalTok{Walc)}
\CommentTok{\# df$freetime=factor(df$freetime)}
\CommentTok{\# df$Medu=factor(df$Medu)}
\CommentTok{\# df$Fedu=factor(df$Fedu)}
\CommentTok{\# df$traveltime=factor(df$traveltime)}
\CommentTok{\# df$studytime=factor(df$studytime)}
\NormalTok{df}\SpecialCharTok{$}\NormalTok{goout}\OtherTok{=}\FunctionTok{factor}\NormalTok{(df}\SpecialCharTok{$}\NormalTok{goout)}
\NormalTok{df}\SpecialCharTok{$}\NormalTok{health}\OtherTok{=}\FunctionTok{factor}\NormalTok{(df}\SpecialCharTok{$}\NormalTok{health)}

\DocumentationTok{\#\# On calcule la moyenne des étudiants}
\NormalTok{df}\SpecialCharTok{$}\NormalTok{Moy }\OtherTok{=}\NormalTok{ (df}\SpecialCharTok{$}\NormalTok{G1}\SpecialCharTok{+}\NormalTok{df}\SpecialCharTok{$}\NormalTok{G2}\SpecialCharTok{+}\NormalTok{df}\SpecialCharTok{$}\NormalTok{G3)}\SpecialCharTok{/}\DecValTok{3}

\DocumentationTok{\#\# On rajoute la réussite scolaire comme variable qualitative que nous devrons prédire.}
\CommentTok{\# df$RS = factor(df$Moy\textgreater{}=10)}
\NormalTok{df}\SpecialCharTok{$}\NormalTok{RS }\OtherTok{=} \StringTok{"admis"}
\CommentTok{\# df$RS[df$Moy\textless{}10]="admis par conseil"}
\NormalTok{df}\SpecialCharTok{$}\NormalTok{RS[df}\SpecialCharTok{$}\NormalTok{Moy}\SpecialCharTok{\textless{}}\DecValTok{10}\NormalTok{]}\OtherTok{=}\StringTok{"redoublement"}
\NormalTok{df}\SpecialCharTok{$}\NormalTok{RS[df}\SpecialCharTok{$}\NormalTok{Moy}\SpecialCharTok{\textless{}}\FloatTok{8.50}\NormalTok{]}\OtherTok{=}\StringTok{"exclusion"}
\NormalTok{df}\SpecialCharTok{$}\NormalTok{RS}\OtherTok{=}\FunctionTok{factor}\NormalTok{(df}\SpecialCharTok{$}\NormalTok{RS)}
\FunctionTok{head}\NormalTok{(df)}
\end{Highlighting}
\end{Shaded}

\begin{verbatim}
##   school sex age address famsize Pstatus Medu Fedu     Mjob     Fjob     reason
## 1     GP   F  18       U     GT3       A    4    4  at_home  teacher     course
## 2     GP   F  17       U     GT3       T    1    1  at_home    other     course
## 3     GP   F  15       U     LE3       T    1    1  at_home    other      other
## 4     GP   F  15       U     GT3       T    4    2   health services       home
## 5     GP   F  16       U     GT3       T    3    3    other    other       home
## 6     GP   M  16       U     LE3       T    4    3 services    other reputation
##   guardian traveltime studytime failures schoolsup famsup paid activities
## 1   mother          2         2        0       yes     no   no         no
## 2   father          1         2        0        no    yes   no         no
## 3   mother          1         2        3       yes     no  yes         no
## 4   mother          1         3        0        no    yes  yes        yes
## 5   father          1         2        0        no    yes  yes         no
## 6   mother          1         2        0        no    yes  yes        yes
##   nursery higher internet romantic famrel freetime goout Dalc Walc health
## 1     yes    yes       no       no      4        3     4    1    1      3
## 2      no    yes      yes       no      5        3     3    1    1      3
## 3     yes    yes      yes       no      4        3     2    2    3      3
## 4     yes    yes      yes      yes      3        2     2    1    1      5
## 5     yes    yes       no       no      4        3     2    1    2      5
## 6     yes    yes      yes       no      5        4     2    1    2      5
##   absences G1 G2 G3       Moy           RS
## 1        6  5  6  6  5.666667    exclusion
## 2        4  5  5  6  5.333333    exclusion
## 3       10  7  8 10  8.333333    exclusion
## 4        2 15 14 15 14.666667        admis
## 5        4  6 10 10  8.666667 redoublement
## 6       10 15 15 15 15.000000        admis
\end{verbatim}

\hypertarget{exploration-des-donnuxe9es-uxe9tudes-des-variables}{%
\section{3. Exploration des données : études des
variables}\label{exploration-des-donnuxe9es-uxe9tudes-des-variables}}

Cette partie consiste à appliquer des méthodes de statistiques
descriptives afin de mieux comprendre le jeu de données. On se concentre
sur l'analyse de la distribution des variables et leur corrélation avec
les résultats scolaires.

\hypertarget{les-variables-qualitatives}{%
\subsection{Les variables
qualitatives}\label{les-variables-qualitatives}}

\hypertarget{le-sexe-des-uxe9tudiants}{%
\subsubsection{Le sexe des étudiants}\label{le-sexe-des-uxe9tudiants}}

D'après le diagramme, le dataset est plutôt équilibré en terme d'hommes
et de femmes,il y même plus de femmes que d'hommes dans ce lycées. On
étudie ensuite le lien entre le sexe et les notes en effectant une
ANOVA1. D'après le test de Fisher, p-value \textgreater{} 5\% donc il
n'y a pas d'effet du sexe sur les notes. D'après le test d'indépendances
de Chi2 avec l'admission, le sexe des élèves n'a pas de lien avec leur
réussite scolaire.

\begin{Shaded}
\begin{Highlighting}[]
\CommentTok{\# Distribution}
\FunctionTok{library}\NormalTok{(ggplot2)}
\end{Highlighting}
\end{Shaded}

\begin{verbatim}
## 
## Attachement du package : 'ggplot2'
\end{verbatim}

\begin{verbatim}
## L'objet suivant est masqué depuis 'package:randomForest':
## 
##     margin
\end{verbatim}

\begin{Shaded}
\begin{Highlighting}[]
\FunctionTok{ggplot}\NormalTok{(df, }\FunctionTok{aes}\NormalTok{(}\AttributeTok{x =}\NormalTok{ sex)) }\SpecialCharTok{+} 
  \FunctionTok{geom\_bar}\NormalTok{(}\AttributeTok{fill =} \StringTok{"steelblue"}\NormalTok{) }\SpecialCharTok{+} 
  \FunctionTok{labs}\NormalTok{(}\AttributeTok{title =} \StringTok{"Répartition des sexes"}\NormalTok{)}
\end{Highlighting}
\end{Shaded}

\includegraphics{projet_files/figure-latex/unnamed-chunk-3-1.pdf}

\begin{Shaded}
\begin{Highlighting}[]
\CommentTok{\# Lien avec la moyenne}
\FunctionTok{summary}\NormalTok{(}\FunctionTok{lm}\NormalTok{(Moy }\SpecialCharTok{\textasciitilde{}}\NormalTok{ sex,}\AttributeTok{data=}\NormalTok{df))}
\end{Highlighting}
\end{Shaded}

\begin{verbatim}
## 
## Call:
## lm(formula = Moy ~ sex, data = df)
## 
## Residuals:
##      Min       1Q   Median       3Q      Max 
## -10.0152  -2.0152  -0.0152   2.1722   8.1722 
## 
## Coefficients:
##             Estimate Std. Error t value Pr(>|t|)    
## (Intercept)  11.3486     0.1324  85.706   <2e-16 ***
## sexM         -0.1874     0.2010  -0.932    0.351    
## ---
## Signif. codes:  0 '***' 0.001 '**' 0.01 '*' 0.05 '.' 0.1 ' ' 1
## 
## Residual standard error: 3.219 on 1042 degrees of freedom
## Multiple R-squared:  0.0008335,  Adjusted R-squared:  -0.0001254 
## F-statistic: 0.8693 on 1 and 1042 DF,  p-value: 0.3514
\end{verbatim}

\begin{Shaded}
\begin{Highlighting}[]
\CommentTok{\# Lien avec la réussite}
\FunctionTok{chisq.test}\NormalTok{(df}\SpecialCharTok{$}\NormalTok{sex,df}\SpecialCharTok{$}\NormalTok{RS)}
\end{Highlighting}
\end{Shaded}

\begin{verbatim}
## 
##  Pearson's Chi-squared test
## 
## data:  df$sex and df$RS
## X-squared = 1.1035, df = 2, p-value = 0.5759
\end{verbatim}

\hypertarget{la-taille-de-la-famille}{%
\subsubsection{La taille de la famille}\label{la-taille-de-la-famille}}

On a deux fois plus de grandes familles que de petites familles. D'après
le test de Fisher, il y a bien un impactde taille de la famille sur les
notes. Le test d'indépendance avec la réussite indique cependant que la
taille de la famille n'est pas liée à la réussite scolaire.

\begin{Shaded}
\begin{Highlighting}[]
\CommentTok{\# Distribution}
\FunctionTok{ggplot}\NormalTok{(}\AttributeTok{data =}\NormalTok{ df, }\FunctionTok{aes}\NormalTok{(}\AttributeTok{x =}\NormalTok{ famsize, }\AttributeTok{fill =}\NormalTok{ famsize)) }\SpecialCharTok{+}
  \FunctionTok{geom\_bar}\NormalTok{() }\SpecialCharTok{+}
  \FunctionTok{labs}\NormalTok{(}\AttributeTok{title =} \StringTok{"Distribution de la taille de la famille"}\NormalTok{,}
       \AttributeTok{x =} \StringTok{"Taille de la famille"}\NormalTok{, }\AttributeTok{y =} \StringTok{"Nombre d\textquotesingle{}étudiants"}\NormalTok{) }\SpecialCharTok{+}
  \FunctionTok{scale\_fill\_manual}\NormalTok{(}\AttributeTok{values =} \FunctionTok{c}\NormalTok{(}\StringTok{"\#7570b3"}\NormalTok{,}\StringTok{"\#F0E442"}\NormalTok{))}
\end{Highlighting}
\end{Shaded}

\includegraphics{projet_files/figure-latex/unnamed-chunk-4-1.pdf}

\begin{Shaded}
\begin{Highlighting}[]
\CommentTok{\# Lien avec les notes}
\FunctionTok{summary}\NormalTok{(}\FunctionTok{lm}\NormalTok{(Moy }\SpecialCharTok{\textasciitilde{}}\NormalTok{ famsize,}\AttributeTok{data=}\NormalTok{df))}
\end{Highlighting}
\end{Shaded}

\begin{verbatim}
## 
## Call:
## lm(formula = Moy ~ famsize, data = df)
## 
## Residuals:
##     Min      1Q  Median      3Q     Max 
## -9.9096 -1.9096 -0.1391  2.1942  8.1942 
## 
## Coefficients:
##             Estimate Std. Error t value Pr(>|t|)    
## (Intercept)  11.1391     0.1183   94.15   <2e-16 ***
## famsizeLE3    0.4371     0.2185    2.00   0.0457 *  
## ---
## Signif. codes:  0 '***' 0.001 '**' 0.01 '*' 0.05 '.' 0.1 ' ' 1
## 
## Residual standard error: 3.214 on 1042 degrees of freedom
## Multiple R-squared:  0.003825,   Adjusted R-squared:  0.002869 
## F-statistic: 4.001 on 1 and 1042 DF,  p-value: 0.04573
\end{verbatim}

\begin{Shaded}
\begin{Highlighting}[]
\CommentTok{\# Lien avec la réussite}
\FunctionTok{chisq.test}\NormalTok{(df}\SpecialCharTok{$}\NormalTok{famsize,df}\SpecialCharTok{$}\NormalTok{RS)}
\end{Highlighting}
\end{Shaded}

\begin{verbatim}
## 
##  Pearson's Chi-squared test
## 
## data:  df$famsize and df$RS
## X-squared = 4.5986, df = 2, p-value = 0.1003
\end{verbatim}

\hypertarget{situation-familliale-suxe9paration-des-parents}{%
\subsubsection{Situation familliale : séparation des
parents}\label{situation-familliale-suxe9paration-des-parents}}

Le jeu est très déséquilibré au sujet de la situation famille : il y a 4
fois plus d'étudiants qui ont leurs parents qui vivent ensemble. De
plus, le test de Fisher indique que la situation familliale n'a pas
d'impact sur les notes. Le test de Chi2 soutient que le status des
parents et la réussite scolaire sont indépendants.

\begin{Shaded}
\begin{Highlighting}[]
\CommentTok{\# Distribution}

\FunctionTok{ggplot}\NormalTok{(}\AttributeTok{data =}\NormalTok{ df, }\FunctionTok{aes}\NormalTok{(}\AttributeTok{x =}\NormalTok{ Pstatus, }\AttributeTok{fill =}\NormalTok{ Pstatus)) }\SpecialCharTok{+}
  \FunctionTok{geom\_bar}\NormalTok{() }\SpecialCharTok{+}
  \FunctionTok{scale\_fill\_manual}\NormalTok{(}\AttributeTok{values =} \FunctionTok{c}\NormalTok{(}\StringTok{"\#0072B2"}\NormalTok{, }\StringTok{"\#009E73"}\NormalTok{)) }\SpecialCharTok{+}
  \FunctionTok{labs}\NormalTok{(}\AttributeTok{title =} \StringTok{"Distribution de la situation conjugale des parents"}\NormalTok{,}
       \AttributeTok{x =} \StringTok{"Situation conjugale"}\NormalTok{, }\AttributeTok{y =} \StringTok{"Nombre d\textquotesingle{}étudiants"}\NormalTok{)}
\end{Highlighting}
\end{Shaded}

\includegraphics{projet_files/figure-latex/unnamed-chunk-5-1.pdf}

\begin{Shaded}
\begin{Highlighting}[]
\CommentTok{\# Lien avec les notes}
\FunctionTok{summary}\NormalTok{(}\FunctionTok{lm}\NormalTok{(Moy }\SpecialCharTok{\textasciitilde{}}\NormalTok{ Pstatus,}\AttributeTok{data=}\NormalTok{df))}
\end{Highlighting}
\end{Shaded}

\begin{verbatim}
## 
## Call:
## lm(formula = Moy ~ Pstatus, data = df)
## 
## Residuals:
##      Min       1Q   Median       3Q      Max 
## -10.0744  -1.9155   0.0845   2.0845   8.0845 
## 
## Coefficients:
##             Estimate Std. Error t value Pr(>|t|)    
## (Intercept)  11.4077     0.2927   38.97   <2e-16 ***
## PstatusT     -0.1589     0.3113   -0.51     0.61    
## ---
## Signif. codes:  0 '***' 0.001 '**' 0.01 '*' 0.05 '.' 0.1 ' ' 1
## 
## Residual standard error: 3.22 on 1042 degrees of freedom
## Multiple R-squared:  0.0002499,  Adjusted R-squared:  -0.0007095 
## F-statistic: 0.2605 on 1 and 1042 DF,  p-value: 0.6099
\end{verbatim}

\begin{Shaded}
\begin{Highlighting}[]
\CommentTok{\# Lien avec la réussite}
\FunctionTok{chisq.test}\NormalTok{(df}\SpecialCharTok{$}\NormalTok{Pstatus,df}\SpecialCharTok{$}\NormalTok{RS)}
\end{Highlighting}
\end{Shaded}

\begin{verbatim}
## 
##  Pearson's Chi-squared test
## 
## data:  df$Pstatus and df$RS
## X-squared = 0.9565, df = 2, p-value = 0.6199
\end{verbatim}

\hypertarget{travail-des-parents}{%
\subsubsection{Travail des parents}\label{travail-des-parents}}

Dans les deux cas, others et services sont les catégories qui dominent.
Une différence notable est la que la proportion de femme au-foyer est
bien plus élevée que celle des hommes. D'après le test de Fisher, le
travail de la mère a un impact sur les notes, contrairement à celui du
père. Les résultats des test de Chis2 suivent les résultats des test de
Fisher : le travail de la mère et la réussite scolaire sont bien
corrélés mais celui du père n'a pas d'impact.

\begin{Shaded}
\begin{Highlighting}[]
\CommentTok{\#Distributions}
\NormalTok{g2}\OtherTok{=}\FunctionTok{ggplot}\NormalTok{(}\AttributeTok{data =}\NormalTok{ df, }\FunctionTok{aes}\NormalTok{(}\AttributeTok{x =}\NormalTok{ Mjob)) }\SpecialCharTok{+}
  \FunctionTok{geom\_bar}\NormalTok{() }\SpecialCharTok{+}
  \FunctionTok{labs}\NormalTok{(}\AttributeTok{title =} \StringTok{"Distribution du travail de la mère"}\NormalTok{) }\SpecialCharTok{+}
  \FunctionTok{scale\_fill\_manual}\NormalTok{(}\AttributeTok{values =} \FunctionTok{c}\NormalTok{(}\StringTok{"\#7570b3"}\NormalTok{,}\StringTok{"\#0072B2"}\NormalTok{, }\StringTok{"\#E69F00"}\NormalTok{, }\StringTok{"\#009E73"}\NormalTok{, }\StringTok{"\#F0E442"}\NormalTok{))}

\NormalTok{g1}\OtherTok{=}\FunctionTok{ggplot}\NormalTok{(}\AttributeTok{data =}\NormalTok{ df, }\FunctionTok{aes}\NormalTok{(}\AttributeTok{x =}\NormalTok{ Fjob)) }\SpecialCharTok{+}
  \FunctionTok{geom\_bar}\NormalTok{() }\SpecialCharTok{+}
  \FunctionTok{labs}\NormalTok{(}\AttributeTok{title=}\StringTok{"Distribution du travail du père"}\NormalTok{) }\SpecialCharTok{+}
  \FunctionTok{scale\_fill\_manual}\NormalTok{(}\AttributeTok{values =} \FunctionTok{c}\NormalTok{(}\StringTok{"\#7570b3"}\NormalTok{,}\StringTok{"\#0072B2"}\NormalTok{, }\StringTok{"\#E69F00"}\NormalTok{, }\StringTok{"\#009E73"}\NormalTok{, }\StringTok{"\#F0E442"}\NormalTok{))}

\FunctionTok{grid.arrange}\NormalTok{(g1, g2, }\AttributeTok{ncol =} \DecValTok{2}\NormalTok{)}
\end{Highlighting}
\end{Shaded}

\includegraphics{projet_files/figure-latex/unnamed-chunk-6-1.pdf}

\begin{Shaded}
\begin{Highlighting}[]
\CommentTok{\# Lien avec les notes}
\FunctionTok{summary}\NormalTok{(}\FunctionTok{lm}\NormalTok{(Moy }\SpecialCharTok{\textasciitilde{}}\NormalTok{ Medu}\SpecialCharTok{+}\NormalTok{Fedu,}\AttributeTok{data=}\NormalTok{df))}
\end{Highlighting}
\end{Shaded}

\begin{verbatim}
## 
## Call:
## lm(formula = Moy ~ Medu + Fedu, data = df)
## 
## Residuals:
##     Min      1Q  Median      3Q     Max 
## -10.265  -1.732   0.068   2.126   7.852 
## 
## Coefficients:
##             Estimate Std. Error t value Pr(>|t|)    
## (Intercept)   9.4233     0.2595  36.316  < 2e-16 ***
## Medu          0.5214     0.1125   4.635 4.02e-06 ***
## Fedu          0.2037     0.1150   1.771   0.0769 .  
## ---
## Signif. codes:  0 '***' 0.001 '**' 0.01 '*' 0.05 '.' 0.1 ' ' 1
## 
## Residual standard error: 3.133 on 1041 degrees of freedom
## Multiple R-squared:  0.05434,    Adjusted R-squared:  0.05252 
## F-statistic: 29.91 on 2 and 1041 DF,  p-value: 2.344e-13
\end{verbatim}

\begin{Shaded}
\begin{Highlighting}[]
\CommentTok{\# Lien avec la réussite}
\FunctionTok{chisq.test}\NormalTok{(df}\SpecialCharTok{$}\NormalTok{Mjob,df}\SpecialCharTok{$}\NormalTok{RS)}
\end{Highlighting}
\end{Shaded}

\begin{verbatim}
## 
##  Pearson's Chi-squared test
## 
## data:  df$Mjob and df$RS
## X-squared = 21.736, df = 8, p-value = 0.005429
\end{verbatim}

\begin{Shaded}
\begin{Highlighting}[]
\FunctionTok{chisq.test}\NormalTok{(df}\SpecialCharTok{$}\NormalTok{Fjob,df}\SpecialCharTok{$}\NormalTok{RS)}
\end{Highlighting}
\end{Shaded}

\begin{verbatim}
## 
##  Pearson's Chi-squared test
## 
## data:  df$Fjob and df$RS
## X-squared = 7.4964, df = 8, p-value = 0.4841
\end{verbatim}

\begin{Shaded}
\begin{Highlighting}[]
\CommentTok{\# AFC sur le travail de la mère}
\NormalTok{df.Mjob }\OtherTok{=} \FunctionTok{data.frame}\NormalTok{(df}\SpecialCharTok{$}\NormalTok{Mjob,df}\SpecialCharTok{$}\NormalTok{RS)}
\NormalTok{table.Mjob }\OtherTok{=} \FunctionTok{table}\NormalTok{(df.Mjob)}
\NormalTok{res }\OtherTok{=} \FunctionTok{CA}\NormalTok{(table.Mjob)}
\end{Highlighting}
\end{Shaded}

\includegraphics{projet_files/figure-latex/unnamed-chunk-6-2.pdf} \#\#\#
Les raisons du choix d'école

D'après le digramme circulaire, seule ``other'' possède un petit
effectif alors que ``course'' domine. Ainsi, les élèves vont
majoritairement en cours car ils les apprécient. D'après l'ANOVA1, il
est clair que la raison d'aller en cours impacte les notes des étudiants
(p-value \textless{} 5\%). Cela paraît cohérent étant donné que cela
détermine leur motivation à avoir de bonnes notes. De la même manière,
la raison est bien corrélé avec la réussite scolaire, ce qui parâit bien
cohérent.

\begin{Shaded}
\begin{Highlighting}[]
\CommentTok{\# Distribution}
\FunctionTok{ggplot}\NormalTok{(}\AttributeTok{data =}\NormalTok{ df, }\FunctionTok{aes}\NormalTok{(}\AttributeTok{x =}\NormalTok{ reason, }\AttributeTok{fill =}\NormalTok{ reason)) }\SpecialCharTok{+}
  \FunctionTok{geom\_bar}\NormalTok{() }\SpecialCharTok{+}
  \FunctionTok{labs}\NormalTok{(}\AttributeTok{title=}\StringTok{"Distribution du travail du père"}\NormalTok{) }\SpecialCharTok{+}
  \FunctionTok{scale\_fill\_manual}\NormalTok{(}\AttributeTok{values =} \FunctionTok{c}\NormalTok{(}\StringTok{"\#7570b3"}\NormalTok{,}\StringTok{"\#0072B2"}\NormalTok{, }\StringTok{"\#E69F00"}\NormalTok{, }\StringTok{"\#009E73"}\NormalTok{, }\StringTok{"\#F0E442"}\NormalTok{))}
\end{Highlighting}
\end{Shaded}

\includegraphics{projet_files/figure-latex/unnamed-chunk-7-1.pdf}

\begin{Shaded}
\begin{Highlighting}[]
\CommentTok{\# Lien avec les notes}
\FunctionTok{summary}\NormalTok{(}\FunctionTok{lm}\NormalTok{(Moy}\SpecialCharTok{\textasciitilde{}}\NormalTok{ reason,}\AttributeTok{data=}\NormalTok{df))}
\end{Highlighting}
\end{Shaded}

\begin{verbatim}
## 
## Call:
## lm(formula = Moy ~ reason, data = df)
## 
## Residuals:
##      Min       1Q   Median       3Q      Max 
## -10.3858  -1.8791  -0.0052   2.1209   7.7876 
## 
## Coefficients:
##                  Estimate Std. Error t value Pr(>|t|)    
## (Intercept)      10.87907    0.15372  70.771  < 2e-16 ***
## reasonhome        0.45943    0.25103   1.830   0.0675 .  
## reasonother      -0.03956    0.34309  -0.115   0.9082    
## reasonreputation  1.17335    0.25417   4.616 4.39e-06 ***
## ---
## Signif. codes:  0 '***' 0.001 '**' 0.01 '*' 0.05 '.' 0.1 ' ' 1
## 
## Residual standard error: 3.188 on 1040 degrees of freedom
## Multiple R-squared:  0.02209,    Adjusted R-squared:  0.01927 
## F-statistic: 7.832 on 3 and 1040 DF,  p-value: 3.587e-05
\end{verbatim}

\begin{Shaded}
\begin{Highlighting}[]
\CommentTok{\# Lien avec la réussite}
\FunctionTok{chisq.test}\NormalTok{(df}\SpecialCharTok{$}\NormalTok{reason,df}\SpecialCharTok{$}\NormalTok{RS)}
\end{Highlighting}
\end{Shaded}

\begin{verbatim}
## 
##  Pearson's Chi-squared test
## 
## data:  df$reason and df$RS
## X-squared = 15.479, df = 6, p-value = 0.01684
\end{verbatim}

\begin{Shaded}
\begin{Highlighting}[]
\CommentTok{\# AFC sur le travail de la mère}
\NormalTok{df.reason }\OtherTok{=} \FunctionTok{data.frame}\NormalTok{(df}\SpecialCharTok{$}\NormalTok{reason,df}\SpecialCharTok{$}\NormalTok{RS)}
\NormalTok{table.reason }\OtherTok{=} \FunctionTok{table}\NormalTok{(df.reason)}
\NormalTok{res }\OtherTok{=} \FunctionTok{CA}\NormalTok{(table.reason)}
\end{Highlighting}
\end{Shaded}

\includegraphics{projet_files/figure-latex/unnamed-chunk-7-2.pdf} On
voit bien avec l'AFC que les personnes étant admises sont celles qui
choisissent l'école pour sa réputation et sa proximité par rapport à
leur domicile. A l'inverse on voit que les étudiants qui ont échoués
sont ceux qui ont choisis l'école pour les cours ou d'autres raisons. On
voit ici une des limite de cette méthode, en effet, on peut penser que
les élèves qui réussisent le mieux sont ceux qui sont le plus motivés et
donc qui ont chosies l'école pour les cours plus que pour sa
réputation.\\
\#\#\# Les relations

Il y a environ deux fois plus de jeunes célibataires que de jeunes en
couple. On peut penser qu'être en couple réduit le temps passé à étudier
et rajoute des distractions, donc il devrait avoir un impact négatif sur
les notes. D'après le test de Fisher, la p-value est fortement
inférieure à 5\%, donc on rejette H0: il y a bien un lien entre
situation romantique et notes, ce qui rejoint bien l'idée de départ. Il
serait donc intéréssant d'étudier la distribution des notes selon la
situation romantique. D'après les boxplots, les différences sont assez
minimes, même si on peut aperçevoir que les notes des célibataires sont
légèrement meilleures. Cependant, la présence de relation amoureuse n'a
pas d'impact sur la réussite scolaire. Ainsi, être en couple fait
baisser la moyenne mais n'est pas un facteur d'échec.

\begin{Shaded}
\begin{Highlighting}[]
\CommentTok{\# Distribution}
\NormalTok{gr1}\OtherTok{=}\FunctionTok{ggplot}\NormalTok{(df, }\FunctionTok{aes}\NormalTok{(}\AttributeTok{x =}\NormalTok{ romantic)) }\SpecialCharTok{+}
  \FunctionTok{geom\_bar}\NormalTok{(}\AttributeTok{fill =} \StringTok{"steelblue"}\NormalTok{) }\SpecialCharTok{+}
  \FunctionTok{labs}\NormalTok{(}\AttributeTok{title =} \StringTok{"Distribution des personnes en couple"}\NormalTok{,}
       \AttributeTok{x =} \StringTok{"Couple"}\NormalTok{, }\AttributeTok{y =} \StringTok{"Nombre d\textquotesingle{}étudiants"}\NormalTok{)}

\CommentTok{\# Lien avec les notes}
\FunctionTok{summary}\NormalTok{(}\FunctionTok{lm}\NormalTok{(Moy}\SpecialCharTok{\textasciitilde{}}\NormalTok{ romantic,}\AttributeTok{data=}\NormalTok{df))}
\end{Highlighting}
\end{Shaded}

\begin{verbatim}
## 
## Call:
## lm(formula = Moy ~ romantic, data = df)
## 
## Residuals:
##      Min       1Q   Median       3Q      Max 
## -10.1486  -1.9455   0.1222   2.1847   7.8514 
## 
## Coefficients:
##             Estimate Std. Error t value Pr(>|t|)    
## (Intercept)  11.4819     0.1236  92.871  < 2e-16 ***
## romanticyes  -0.6041     0.2074  -2.913  0.00366 ** 
## ---
## Signif. codes:  0 '***' 0.001 '**' 0.01 '*' 0.05 '.' 0.1 ' ' 1
## 
## Residual standard error: 3.207 on 1042 degrees of freedom
## Multiple R-squared:  0.008077,   Adjusted R-squared:  0.007125 
## F-statistic: 8.485 on 1 and 1042 DF,  p-value: 0.003658
\end{verbatim}

\begin{Shaded}
\begin{Highlighting}[]
\NormalTok{yes }\OtherTok{=}\NormalTok{ df}\SpecialCharTok{$}\NormalTok{Moy[df}\SpecialCharTok{$}\NormalTok{romantic}\SpecialCharTok{==}\StringTok{\textquotesingle{}yes\textquotesingle{}}\NormalTok{]}
\NormalTok{no }\OtherTok{=}\NormalTok{ df}\SpecialCharTok{$}\NormalTok{Moy[df}\SpecialCharTok{$}\NormalTok{romantic}\SpecialCharTok{==}\StringTok{\textquotesingle{}no\textquotesingle{}}\NormalTok{]}

\CommentTok{\# Boxplot des notes}
\NormalTok{gr2}\OtherTok{=}\FunctionTok{ggplot}\NormalTok{(}\AttributeTok{data =}\NormalTok{ df, }\FunctionTok{aes}\NormalTok{(}\AttributeTok{x =}\NormalTok{ romantic, }\AttributeTok{y =}\NormalTok{ Moy, }\AttributeTok{fill =}\NormalTok{ romantic)) }\SpecialCharTok{+}
  \FunctionTok{geom\_boxplot}\NormalTok{() }\SpecialCharTok{+}
  \FunctionTok{scale\_fill\_manual}\NormalTok{(}\AttributeTok{values =} \FunctionTok{c}\NormalTok{(}\StringTok{"\#0072B2"}\NormalTok{, }\StringTok{"\#F0E442"}\NormalTok{)) }\SpecialCharTok{+}
  \FunctionTok{labs}\NormalTok{(}\AttributeTok{title =} \StringTok{"Relation entre être en couple et les notes"}\NormalTok{,}
       \AttributeTok{x =} \StringTok{"Couple"}\NormalTok{, }\AttributeTok{y =} \StringTok{"Notes"}\NormalTok{)}

\FunctionTok{grid.arrange}\NormalTok{(gr1, gr2, }\AttributeTok{ncol =} \DecValTok{2}\NormalTok{)}
\end{Highlighting}
\end{Shaded}

\includegraphics{projet_files/figure-latex/unnamed-chunk-8-1.pdf}

\begin{Shaded}
\begin{Highlighting}[]
\CommentTok{\# Lien avec la réussite}
\FunctionTok{chisq.test}\NormalTok{(df}\SpecialCharTok{$}\NormalTok{romantic,df}\SpecialCharTok{$}\NormalTok{RS)}
\end{Highlighting}
\end{Shaded}

\begin{verbatim}
## 
##  Pearson's Chi-squared test
## 
## data:  df$romantic and df$RS
## X-squared = 5.5477, df = 2, p-value = 0.06242
\end{verbatim}

\hypertarget{volontuxe9-de-faire-des-uxe9tudes-supuxe9rieures}{%
\subsubsection{Volonté de faire des études
supérieures}\label{volontuxe9-de-faire-des-uxe9tudes-supuxe9rieures}}

On observe qu'au moins 80\% des élèves veulent continuer leur études
après le lycée, ce qui est plutôt rassurant. De plus, d'après le test de
Fisher, les deux variables sont corrélées. On peut également annoncer
que ceux qui veulent faire des études supérieures tendent à avoir de
meilleures notes grâce au test unilatéral. A priori, la volonté de faire
des études supérieures est corrélée à la réussite scolaire. Donc, ceux
qui veulent poursuivre leurs études auront de meileures notes et
tendance à ne pas être en échec.

\begin{Shaded}
\begin{Highlighting}[]
\CommentTok{\# distribution}
\NormalTok{g1}\OtherTok{=}\FunctionTok{ggplot}\NormalTok{(df, }\FunctionTok{aes}\NormalTok{(}\AttributeTok{x =}\NormalTok{ higher, }\AttributeTok{fill =}\NormalTok{ higher)) }\SpecialCharTok{+}
  \FunctionTok{geom\_bar}\NormalTok{() }\SpecialCharTok{+}
  \FunctionTok{labs}\NormalTok{(}\AttributeTok{title =} \StringTok{"Distribution de l\textquotesingle{}envie de faire des études supérieures"}\NormalTok{,}
       \AttributeTok{x =} \StringTok{"Envie de faire des études supérieures"}\NormalTok{, }\AttributeTok{y =} \StringTok{"Nombre d\textquotesingle{}étudiants"}\NormalTok{) }\SpecialCharTok{+}
  \FunctionTok{scale\_fill\_manual}\NormalTok{(}\AttributeTok{values =} \FunctionTok{c}\NormalTok{(}\StringTok{"\#E69F00"}\NormalTok{, }\StringTok{"\#0072B2"}\NormalTok{))}

\CommentTok{\# Boxplot des notes en fonction de l\textquotesingle{}envie de faire des études supérieures}
\NormalTok{g2}\OtherTok{=}\FunctionTok{ggplot}\NormalTok{(df, }\FunctionTok{aes}\NormalTok{(}\AttributeTok{x =}\NormalTok{ higher, }\AttributeTok{y =}\NormalTok{ Moy, }\AttributeTok{fill =}\NormalTok{ higher)) }\SpecialCharTok{+}
  \FunctionTok{geom\_boxplot}\NormalTok{() }\SpecialCharTok{+}
  \FunctionTok{labs}\NormalTok{(}\AttributeTok{title =} \StringTok{"Notes en fonction de l\textquotesingle{}envie de faire des études supérieures"}\NormalTok{,}
       \AttributeTok{x =} \StringTok{"Envie de faire des études supérieures"}\NormalTok{, }\AttributeTok{y =} \StringTok{"Moyenne"}\NormalTok{) }\SpecialCharTok{+}
  \FunctionTok{scale\_fill\_manual}\NormalTok{(}\AttributeTok{values =} \FunctionTok{c}\NormalTok{(}\StringTok{"\#E69F00"}\NormalTok{, }\StringTok{"\#0072B2"}\NormalTok{))}

\FunctionTok{grid.arrange}\NormalTok{(g1, g2, }\AttributeTok{ncol =} \DecValTok{2}\NormalTok{)}
\end{Highlighting}
\end{Shaded}

\includegraphics{projet_files/figure-latex/unnamed-chunk-9-1.pdf}

\begin{Shaded}
\begin{Highlighting}[]
\CommentTok{\# Lien avec les notes}
\FunctionTok{summary}\NormalTok{(}\FunctionTok{lm}\NormalTok{(Moy }\SpecialCharTok{\textasciitilde{}}\NormalTok{ higher,}\AttributeTok{data=}\NormalTok{df))}
\end{Highlighting}
\end{Shaded}

\begin{verbatim}
## 
## Call:
## lm(formula = Moy ~ higher, data = df)
## 
## Residuals:
##      Min       1Q   Median       3Q      Max 
## -10.1930  -1.8597   0.1403   2.1403   7.8070 
## 
## Coefficients:
##             Estimate Std. Error t value Pr(>|t|)    
## (Intercept)   8.4869     0.3293  25.775   <2e-16 ***
## higheryes     3.0395     0.3443   8.829   <2e-16 ***
## ---
## Signif. codes:  0 '***' 0.001 '**' 0.01 '*' 0.05 '.' 0.1 ' ' 1
## 
## Residual standard error: 3.106 on 1042 degrees of freedom
## Multiple R-squared:  0.0696, Adjusted R-squared:  0.06871 
## F-statistic: 77.95 on 1 and 1042 DF,  p-value: < 2.2e-16
\end{verbatim}

\begin{Shaded}
\begin{Highlighting}[]
\NormalTok{yes }\OtherTok{=}\NormalTok{ df}\SpecialCharTok{$}\NormalTok{Moy[df}\SpecialCharTok{$}\NormalTok{higher}\SpecialCharTok{==}\StringTok{\textquotesingle{}yes\textquotesingle{}}\NormalTok{]}
\NormalTok{no }\OtherTok{=}\NormalTok{ df}\SpecialCharTok{$}\NormalTok{Moy[df}\SpecialCharTok{$}\NormalTok{higher}\SpecialCharTok{==}\StringTok{\textquotesingle{}no\textquotesingle{}}\NormalTok{]}

\CommentTok{\# Lien avec la réussite}
\FunctionTok{chisq.test}\NormalTok{(df}\SpecialCharTok{$}\NormalTok{higher,df}\SpecialCharTok{$}\NormalTok{RS)}
\end{Highlighting}
\end{Shaded}

\begin{verbatim}
## 
##  Pearson's Chi-squared test
## 
## data:  df$higher and df$RS
## X-squared = 66.594, df = 2, p-value = 3.461e-15
\end{verbatim}

\begin{Shaded}
\begin{Highlighting}[]
\CommentTok{\# AFC sur la volonté de faire des études sup}
\NormalTok{df.higher }\OtherTok{=} \FunctionTok{data.frame}\NormalTok{(df}\SpecialCharTok{$}\NormalTok{higher,df}\SpecialCharTok{$}\NormalTok{RS)}
\NormalTok{table.higher }\OtherTok{=} \FunctionTok{table}\NormalTok{(df.higher)}
\NormalTok{res }\OtherTok{=} \FunctionTok{CA}\NormalTok{(table.higher)}
\end{Highlighting}
\end{Shaded}

\hypertarget{activituxe9s-extrascolaires}{%
\subsubsection{Activités
extrascolaires}\label{activituxe9s-extrascolaires}}

On a autant d'élèves qui pratiquent des activités extrascolaires que
d'élèves qui n'en pratiquent pas, ce qui est plutôt intéréssant. De
plus, le test de Fisher indique plutôt qu'il n'y a pas de liens entre
les activités extrascolaires et les notes, ce qui est plutôt surprenant
étant donné que l'on aurait tendance à penser que les étudiants ayant
des activités, ont moins de temps pour étudier. Dans la même lignée, les
activités sont plutôt indépendates de la réussite d'après le test de
Chi2.

\begin{Shaded}
\begin{Highlighting}[]
\CommentTok{\# Distribution}
\FunctionTok{ggplot}\NormalTok{(df, }\FunctionTok{aes}\NormalTok{(}\AttributeTok{x =}\NormalTok{ activities, }\AttributeTok{fill =}\NormalTok{ activities)) }\SpecialCharTok{+}
  \FunctionTok{geom\_bar}\NormalTok{() }\SpecialCharTok{+}
  \FunctionTok{labs}\NormalTok{(}\AttributeTok{title =} \StringTok{"Distribution de de la pratique d\textquotesingle{}activités extrascolaires"}\NormalTok{,}
       \AttributeTok{x =} \StringTok{"Envisagez{-}vous de faire des études supérieures ?"}\NormalTok{, }
       \AttributeTok{y =} \StringTok{"Nombre d\textquotesingle{}étudiants"}\NormalTok{) }\SpecialCharTok{+}
  \FunctionTok{scale\_fill\_manual}\NormalTok{(}\AttributeTok{values =} \FunctionTok{c}\NormalTok{(}\StringTok{"\#0072B2"}\NormalTok{, }\StringTok{"\#F0E442"}\NormalTok{))}
\end{Highlighting}
\end{Shaded}

\includegraphics{projet_files/figure-latex/unnamed-chunk-10-1.pdf}

\begin{Shaded}
\begin{Highlighting}[]
\CommentTok{\# Lien avec les notes}
\FunctionTok{summary}\NormalTok{(}\FunctionTok{lm}\NormalTok{(Moy }\SpecialCharTok{\textasciitilde{}}\NormalTok{ activities,}\AttributeTok{data=}\NormalTok{df))}
\end{Highlighting}
\end{Shaded}

\begin{verbatim}
## 
## Call:
## lm(formula = Moy ~ activities, data = df)
## 
## Residuals:
##      Min       1Q   Median       3Q      Max 
## -10.1085  -2.0966  -0.0966   2.2248   7.8915 
## 
## Coefficients:
##               Estimate Std. Error t value Pr(>|t|)    
## (Intercept)    11.0966     0.1399  79.292   <2e-16 ***
## activitiesyes   0.3453     0.1991   1.734   0.0831 .  
## ---
## Signif. codes:  0 '***' 0.001 '**' 0.01 '*' 0.05 '.' 0.1 ' ' 1
## 
## Residual standard error: 3.216 on 1042 degrees of freedom
## Multiple R-squared:  0.002879,   Adjusted R-squared:  0.001922 
## F-statistic: 3.008 on 1 and 1042 DF,  p-value: 0.08313
\end{verbatim}

\begin{Shaded}
\begin{Highlighting}[]
\CommentTok{\# Lien avec la réussite}
\FunctionTok{chisq.test}\NormalTok{(df}\SpecialCharTok{$}\NormalTok{activities,df}\SpecialCharTok{$}\NormalTok{RS)}
\end{Highlighting}
\end{Shaded}

\begin{verbatim}
## 
##  Pearson's Chi-squared test
## 
## data:  df$activities and df$RS
## X-squared = 2.5236, df = 2, p-value = 0.2831
\end{verbatim}

\hypertarget{cours-suppluxe9mentaires}{%
\subsubsection{Cours supplémentaires}\label{cours-suppluxe9mentaires}}

Il y a bien plus d'élèves qui ne suivent pas de cours supplémentaires
que d'élèves qui en suivent. Cett distributution est cohérente avec
l'idée qu'on oeut se faire. Le test de Fisher indique plutôt que les
suivis de cours supplémentaires n'a pas d'impact sur la moyenne. De
même, le suivi de cours supplémentaire n'est pas lié à la réussite.

\begin{Shaded}
\begin{Highlighting}[]
\CommentTok{\# Distribution}
\FunctionTok{ggplot}\NormalTok{(}\AttributeTok{data =}\NormalTok{ df, }\FunctionTok{aes}\NormalTok{(}\AttributeTok{x =}\NormalTok{paid, }\AttributeTok{fill =}\NormalTok{ paid)) }\SpecialCharTok{+}
  \FunctionTok{geom\_bar}\NormalTok{() }\SpecialCharTok{+}
  \FunctionTok{labs}\NormalTok{(}\AttributeTok{title =} \StringTok{"Distribution de la pratique des cours supplémentaire"}\NormalTok{,}
       \AttributeTok{x =} \StringTok{"Taille de la famille"}\NormalTok{, }\AttributeTok{y =} \StringTok{"Nombre d\textquotesingle{}étudiants"}\NormalTok{) }\SpecialCharTok{+}
  \FunctionTok{scale\_fill\_manual}\NormalTok{(}\AttributeTok{values =} \FunctionTok{c}\NormalTok{(}\StringTok{"\#7570b3"}\NormalTok{,}\StringTok{"\#F0E442"}\NormalTok{))}
\end{Highlighting}
\end{Shaded}

\includegraphics{projet_files/figure-latex/unnamed-chunk-11-1.pdf}

\begin{Shaded}
\begin{Highlighting}[]
\CommentTok{\# Lien avec la moyenne}
\FunctionTok{summary}\NormalTok{(}\FunctionTok{lm}\NormalTok{(Moy }\SpecialCharTok{\textasciitilde{}}\NormalTok{ paid,}\AttributeTok{data=}\NormalTok{df))}
\end{Highlighting}
\end{Shaded}

\begin{verbatim}
## 
## Call:
## lm(formula = Moy ~ paid, data = df)
## 
## Residuals:
##     Min      1Q  Median      3Q     Max 
## -9.9951 -1.9951  0.0049  2.0049  8.0049 
## 
## Coefficients:
##             Estimate Std. Error t value Pr(>|t|)    
## (Intercept)  11.3285     0.1121  101.05   <2e-16 ***
## paidyes      -0.2906     0.2442   -1.19    0.234    
## ---
## Signif. codes:  0 '***' 0.001 '**' 0.01 '*' 0.05 '.' 0.1 ' ' 1
## 
## Residual standard error: 3.218 on 1042 degrees of freedom
## Multiple R-squared:  0.001357,   Adjusted R-squared:  0.0003986 
## F-statistic: 1.416 on 1 and 1042 DF,  p-value: 0.2344
\end{verbatim}

\begin{Shaded}
\begin{Highlighting}[]
\CommentTok{\# Lien avec la réussite}
\FunctionTok{chisq.test}\NormalTok{(df}\SpecialCharTok{$}\NormalTok{paid,df}\SpecialCharTok{$}\NormalTok{RS)}
\end{Highlighting}
\end{Shaded}

\begin{verbatim}
## 
##  Pearson's Chi-squared test
## 
## data:  df$paid and df$RS
## X-squared = 4.8571, df = 2, p-value = 0.08816
\end{verbatim}

\hypertarget{les-variables-qualitatives-uxe0-modalituxe9s-numuxe9riques}{%
\subsection{Les variables qualitatives à modalités
numériques}\label{les-variables-qualitatives-uxe0-modalituxe9s-numuxe9riques}}

\hypertarget{les-sorties}{%
\subsubsection{Les sorties}\label{les-sorties}}

On remarque que les élèves maintiennent leur vie sociale. La grosse
majorité sont intermédiaires en termes de sorties ce qui est quand même
rassurant. Il y a quand même plus de personnes qui sortent vraiment
beaucoup que de personnes qui ne sortent pas. Le test de Fisher indique
les sorties sont très corrélées au notes et le test de Chi2 montre que
la réussite scolaire est aussi corrélée aux sorties. Ainsi, on retrouve
des résultats qui semblent cohérents et représentatifs de la vie
étudiante.

Etant donné, la corrélation entre RS et goout, on peut effectuer une AFC
pour préciser. On peut remarquer que ceux qui sortent peu-moyennement
auront tendance à être admis alors que ce qui ne sortent pas
(retrait/exclusion social) vont plutôt redoubler et les autres vont
avoir tendances à se faire exclure. On obtient donc des résultats qui
semblent plutôt pertinents.

\begin{Shaded}
\begin{Highlighting}[]
\CommentTok{\# Distribution}
\FunctionTok{ggplot}\NormalTok{(}\AttributeTok{data =}\NormalTok{ df, }\FunctionTok{aes}\NormalTok{(}\AttributeTok{x =}\NormalTok{goout, }\AttributeTok{fill =}\NormalTok{ goout)) }\SpecialCharTok{+}
  \FunctionTok{geom\_bar}\NormalTok{() }\SpecialCharTok{+}
  \FunctionTok{labs}\NormalTok{(}\AttributeTok{title =} \StringTok{"Distribution des sorties"}\NormalTok{,}
       \AttributeTok{x =} \StringTok{"Sorties"}\NormalTok{, }\AttributeTok{y =} \StringTok{"Nombre d\textquotesingle{}étudiants"}\NormalTok{) }
\end{Highlighting}
\end{Shaded}

\includegraphics{projet_files/figure-latex/unnamed-chunk-12-1.pdf}

\begin{Shaded}
\begin{Highlighting}[]
\CommentTok{\# Lien avec la moyenne}
\FunctionTok{summary}\NormalTok{(}\FunctionTok{lm}\NormalTok{(Moy }\SpecialCharTok{\textasciitilde{}}\NormalTok{ goout,}\AttributeTok{data=}\NormalTok{df))}
\end{Highlighting}
\end{Shaded}

\begin{verbatim}
## 
## Call:
## lm(formula = Moy ~ goout, data = df)
## 
## Residuals:
##      Min       1Q   Median       3Q      Max 
## -10.5887  -1.8876  -0.0015   2.1124   7.6652 
## 
## Coefficients:
##             Estimate Std. Error t value Pr(>|t|)    
## (Intercept)  10.5493     0.3770  27.980  < 2e-16 ***
## goout2        1.3727     0.4276   3.210  0.00137 ** 
## goout3        1.0049     0.4151   2.421  0.01564 *  
## goout4        0.4522     0.4320   1.047  0.29548    
## goout5       -0.1853     0.4517  -0.410  0.68178    
## ---
## Signif. codes:  0 '***' 0.001 '**' 0.01 '*' 0.05 '.' 0.1 ' ' 1
## 
## Residual standard error: 3.177 on 1039 degrees of freedom
## Multiple R-squared:  0.02957,    Adjusted R-squared:  0.02583 
## F-statistic: 7.915 on 4 and 1039 DF,  p-value: 2.766e-06
\end{verbatim}

\begin{Shaded}
\begin{Highlighting}[]
\CommentTok{\# Lien avec la réussite}
\FunctionTok{chisq.test}\NormalTok{(df}\SpecialCharTok{$}\NormalTok{goout,df}\SpecialCharTok{$}\NormalTok{RS)}
\end{Highlighting}
\end{Shaded}

\begin{verbatim}
## 
##  Pearson's Chi-squared test
## 
## data:  df$goout and df$RS
## X-squared = 20.537, df = 8, p-value = 0.008485
\end{verbatim}

\begin{Shaded}
\begin{Highlighting}[]
\CommentTok{\# AFC sur les sorties}
\NormalTok{df.goout }\OtherTok{=} \FunctionTok{data.frame}\NormalTok{(df}\SpecialCharTok{$}\NormalTok{goout,df}\SpecialCharTok{$}\NormalTok{RS)}
\NormalTok{table.goout }\OtherTok{=} \FunctionTok{table}\NormalTok{(df.goout)}
\NormalTok{res }\OtherTok{=} \FunctionTok{CA}\NormalTok{(table.goout)}
\end{Highlighting}
\end{Shaded}

\includegraphics{projet_files/figure-latex/unnamed-chunk-12-2.pdf} Avec
un test du \Chit-2 on observe que les variabales goout et RS sont
corrélées (p-valeur petite devant 5\%). Nous allons donc réaliser une
AFC dessus. Egalement la p-valeur associée au test de fisher (sortie de
anova) sur les variables Moy et goout montre que ces grandeurs sont
aussi corrélées.

L'AFC nous montre ici que les étudiants qui sortent raisonnablement sont
ceux qui réussisent le plus. En effet, ceux qui sortent le plus consacre
moins de temps à leur études ce qui peut expliquer ce résultat.
Egalement les étudiants qui ne sortent quasiment pas échouent aussi
beaucoup. Ce manque de sortie peut denoter d'un défaut de
sociabilisation ou des problèmes de santé qui impact gravement la
réussite de l'élève. Le diagramme en baton nous permet de voir que la
majorité des etudiants sortent de manière modéré (modalité 3). L'AFC
nous montre que cela n'est pas un frein à leur réussite.

\hypertarget{la-consommation-dalcool}{%
\subsubsection{La consommation d'alcool}\label{la-consommation-dalcool}}

On s'intérésse enfin à la feature ``principale'' de ce jeu de données,
la consommation d'alcool des étudiants.

\begin{Shaded}
\begin{Highlighting}[]
\CommentTok{\# Distribution}
\FunctionTok{ggplot}\NormalTok{(}\AttributeTok{data =}\NormalTok{ df, }\FunctionTok{aes}\NormalTok{(}\AttributeTok{x =}\NormalTok{Dalc, }\AttributeTok{fill =}\NormalTok{ Dalc)) }\SpecialCharTok{+}
  \FunctionTok{geom\_bar}\NormalTok{() }\SpecialCharTok{+}
  \FunctionTok{labs}\NormalTok{(}\AttributeTok{title =} \StringTok{"Distribution de la consommation d\textquotesingle{}alcool en semaine"}\NormalTok{,}
       \AttributeTok{x =} \StringTok{"Consommation"}\NormalTok{, }\AttributeTok{y =} \StringTok{"Nombre d\textquotesingle{}étudiants"}\NormalTok{) }
\end{Highlighting}
\end{Shaded}

\includegraphics{projet_files/figure-latex/unnamed-chunk-13-1.pdf}

\begin{Shaded}
\begin{Highlighting}[]
\CommentTok{\# Lien avec la moyenne}
\FunctionTok{summary}\NormalTok{(}\FunctionTok{lm}\NormalTok{(Moy }\SpecialCharTok{\textasciitilde{}}\NormalTok{ Dalc,}\AttributeTok{data=}\NormalTok{df))}
\end{Highlighting}
\end{Shaded}

\begin{verbatim}
## 
## Call:
## lm(formula = Moy ~ Dalc, data = df)
## 
## Residuals:
##     Min      1Q  Median      3Q     Max 
## -10.244  -1.911   0.089   2.089   7.756 
## 
## Coefficients:
##             Estimate Std. Error t value Pr(>|t|)    
## (Intercept)  11.5777     0.1181  98.001  < 2e-16 ***
## Dalc2        -0.8889     0.2564  -3.467 0.000547 ***
## Dalc3        -0.8917     0.4013  -2.222 0.026475 *  
## Dalc4        -2.0008     0.6358  -3.147 0.001696 ** 
## Dalc5        -1.3982     0.6358  -2.199 0.028079 *  
## ---
## Signif. codes:  0 '***' 0.001 '**' 0.01 '*' 0.05 '.' 0.1 ' ' 1
## 
## Residual standard error: 3.185 on 1039 degrees of freedom
## Multiple R-squared:  0.02443,    Adjusted R-squared:  0.02068 
## F-statistic: 6.505 on 4 and 1039 DF,  p-value: 3.594e-05
\end{verbatim}

\begin{Shaded}
\begin{Highlighting}[]
\CommentTok{\# Lien avec la réussite}
\FunctionTok{chisq.test}\NormalTok{(df}\SpecialCharTok{$}\NormalTok{Dalc,df}\SpecialCharTok{$}\NormalTok{RS)}
\end{Highlighting}
\end{Shaded}

\begin{verbatim}
## Warning in chisq.test(df$Dalc, df$RS): Chi-squared approximation may be
## incorrect
\end{verbatim}

\begin{verbatim}
## 
##  Pearson's Chi-squared test
## 
## data:  df$Dalc and df$RS
## X-squared = 28.342, df = 8, p-value = 0.0004134
\end{verbatim}

\begin{Shaded}
\begin{Highlighting}[]
\CommentTok{\# AFC sur les sorties}
\NormalTok{df.Dalc }\OtherTok{=} \FunctionTok{data.frame}\NormalTok{(df}\SpecialCharTok{$}\NormalTok{Dalc,df}\SpecialCharTok{$}\NormalTok{RS)}
\NormalTok{table.Dalc }\OtherTok{=} \FunctionTok{table}\NormalTok{(df.Dalc)}
\NormalTok{res }\OtherTok{=} \FunctionTok{CA}\NormalTok{(table.Dalc)}
\end{Highlighting}
\end{Shaded}

\includegraphics{projet_files/figure-latex/unnamed-chunk-13-2.pdf} Avec
le test du \Chi-2 on voit que les variables Dalc et RS sont corrélées.
On va donc realiser une AFC dessus. De même avec la p-valeur du test de
Fisher sur les variables Moy et Dalc, on voit que ces variables sont
aussi corrélées (et c'est logique au vu du test du \Chi-2).

On voit très clairement avec l'AFC que les étudiants qui consomment le
plus d'alcool sont ceux qui réussissent le moins. En effet, une forte
consomation d'alcool témoigne d'un grand nombre de sortie ou bien d'un
grave problème de santé (alcoolisme). Ceux qui réussisent le plus sont
ceux qui consomment le moins d'alcool.

Avec le diagramme en bâton, on voit que la majorité des étudiants ne
consomme quasiment pas d'alcool en semaine. L'AFC montre que cela n'as
pas du tout été un frein pour leur réussite

\begin{Shaded}
\begin{Highlighting}[]
\CommentTok{\# Distribution}
\FunctionTok{ggplot}\NormalTok{(}\AttributeTok{data =}\NormalTok{ df, }\FunctionTok{aes}\NormalTok{(}\AttributeTok{x =}\NormalTok{Walc, }\AttributeTok{fill =}\NormalTok{ Walc)) }\SpecialCharTok{+}
  \FunctionTok{geom\_bar}\NormalTok{() }\SpecialCharTok{+}
  \FunctionTok{labs}\NormalTok{(}\AttributeTok{title =} \StringTok{"Distribution de la consommation d\textquotesingle{}alcool le week{-}end"}\NormalTok{,}
       \AttributeTok{x =} \StringTok{"Consommation"}\NormalTok{, }\AttributeTok{y =} \StringTok{"Nombre d\textquotesingle{}étudiants"}\NormalTok{) }
\end{Highlighting}
\end{Shaded}

\includegraphics{projet_files/figure-latex/unnamed-chunk-14-1.pdf}

\begin{Shaded}
\begin{Highlighting}[]
\CommentTok{\# Lien avec la moyenne}
\FunctionTok{summary}\NormalTok{(}\FunctionTok{lm}\NormalTok{(Moy }\SpecialCharTok{\textasciitilde{}}\NormalTok{ Walc,}\AttributeTok{data=}\NormalTok{df))}
\end{Highlighting}
\end{Shaded}

\begin{verbatim}
## 
## Call:
## lm(formula = Moy ~ Walc, data = df)
## 
## Residuals:
##      Min       1Q   Median       3Q      Max 
## -10.2831  -1.9050   0.0503   2.0614   7.7169 
## 
## Coefficients:
##             Estimate Std. Error t value Pr(>|t|)    
## (Intercept)  11.6164     0.1600  72.593  < 2e-16 ***
## Walc2        -0.1398     0.2626  -0.532 0.594577    
## Walc3        -0.3781     0.2767  -1.366 0.172115    
## Walc4        -1.1768     0.3154  -3.731 0.000201 ***
## Walc5        -1.2831     0.4065  -3.157 0.001642 ** 
## ---
## Signif. codes:  0 '***' 0.001 '**' 0.01 '*' 0.05 '.' 0.1 ' ' 1
## 
## Residual standard error: 3.192 on 1039 degrees of freedom
## Multiple R-squared:  0.0201, Adjusted R-squared:  0.01633 
## F-statistic: 5.328 on 4 and 1039 DF,  p-value: 0.0003004
\end{verbatim}

\begin{Shaded}
\begin{Highlighting}[]
\CommentTok{\# Lien avec la réussite}
\FunctionTok{chisq.test}\NormalTok{(df}\SpecialCharTok{$}\NormalTok{Walc,df}\SpecialCharTok{$}\NormalTok{RS)}
\end{Highlighting}
\end{Shaded}

\begin{verbatim}
## 
##  Pearson's Chi-squared test
## 
## data:  df$Walc and df$RS
## X-squared = 16.5, df = 8, p-value = 0.03576
\end{verbatim}

\begin{Shaded}
\begin{Highlighting}[]
\CommentTok{\# AFC sur les sorties}
\NormalTok{df.Walc }\OtherTok{=} \FunctionTok{data.frame}\NormalTok{(df}\SpecialCharTok{$}\NormalTok{Walc,df}\SpecialCharTok{$}\NormalTok{RS)}
\NormalTok{table.Walc }\OtherTok{=} \FunctionTok{table}\NormalTok{(df.Walc)}
\NormalTok{res }\OtherTok{=} \FunctionTok{CA}\NormalTok{(table.Walc)}
\end{Highlighting}
\end{Shaded}

\includegraphics{projet_files/figure-latex/unnamed-chunk-14-2.pdf} Tout
d'abord on obtiens une p-valeur plus petite que 5\% avec le test du
\Chi-2 ce qui montre que les variables Walc (consomation alcool le week
end) et RS (réussite scoalire) sont corrélées. Nous allons réaliser une
AFC dessus afin de mieux les expliquées. De même avec le test de fisher
(réalisé à l'aide de l'anova) réalisé sur les variables Walc et Moy
montre qu'elles sont corrélées.

De même on obtiens le même résultat avec la consomation d'alcool le week
end (ceux qui consomment le mpoins réussisent le plus), un peu plus
nuancé cependant. En effet, on voit à travers les différents diagrammes
en batons que globalement il y a plus d'étudiants qui consomment de
l'alcool le week end qu'en semaine. On voit donc grâce aux deux AFC que
les étudiants qui consomment plutôt de l'acool le week end réussisent
mieux que les étudiants qui consommentde l'alcool la semaine et le week
end. Ainsi la variable avec la modalité 2 témoigne bien du fait que
consomé de l'alcool en semaine est bien plus néfaste qu'en consomé en
week-end (dans un contexte de soirée).

\hypertarget{les-variables-quantitatives}{%
\subsection{Les variables
quantitatives}\label{les-variables-quantitatives}}

\begin{Shaded}
\begin{Highlighting}[]
\NormalTok{data}\OtherTok{=}\NormalTok{df}
\NormalTok{data\_quanti}\OtherTok{=}\NormalTok{data[}\FunctionTok{c}\NormalTok{(}\DecValTok{3}\NormalTok{,}\DecValTok{7}\NormalTok{,}\DecValTok{8}\NormalTok{,}\DecValTok{13}\NormalTok{,}\DecValTok{14}\NormalTok{,}\DecValTok{15}\NormalTok{,}\DecValTok{25}\NormalTok{,}\DecValTok{26}\NormalTok{,}\DecValTok{27}\NormalTok{,}\DecValTok{28}\NormalTok{,}\DecValTok{29}\NormalTok{,}\DecValTok{30}\NormalTok{,}\DecValTok{31}\NormalTok{,}\DecValTok{32}\NormalTok{,}\DecValTok{33}\NormalTok{)]}
\NormalTok{data\_quanti\_mat}\OtherTok{=}\NormalTok{df.mat[}\FunctionTok{c}\NormalTok{(}\DecValTok{3}\NormalTok{,}\DecValTok{7}\NormalTok{,}\DecValTok{8}\NormalTok{,}\DecValTok{13}\NormalTok{,}\DecValTok{14}\NormalTok{,}\DecValTok{15}\NormalTok{,}\DecValTok{25}\NormalTok{,}\DecValTok{26}\NormalTok{,}\DecValTok{27}\NormalTok{,}\DecValTok{28}\NormalTok{,}\DecValTok{29}\NormalTok{,}\DecValTok{30}\NormalTok{,}\DecValTok{31}\NormalTok{,}\DecValTok{32}\NormalTok{,}\DecValTok{33}\NormalTok{)]}
\NormalTok{data\_quanti\_por}\OtherTok{=}\NormalTok{df.por[}\FunctionTok{c}\NormalTok{(}\DecValTok{3}\NormalTok{,}\DecValTok{7}\NormalTok{,}\DecValTok{8}\NormalTok{,}\DecValTok{13}\NormalTok{,}\DecValTok{14}\NormalTok{,}\DecValTok{15}\NormalTok{,}\DecValTok{25}\NormalTok{,}\DecValTok{26}\NormalTok{,}\DecValTok{27}\NormalTok{,}\DecValTok{28}\NormalTok{,}\DecValTok{29}\NormalTok{,}\DecValTok{30}\NormalTok{,}\DecValTok{31}\NormalTok{,}\DecValTok{32}\NormalTok{,}\DecValTok{33}\NormalTok{)]}
\FunctionTok{head}\NormalTok{(data\_quanti)}
\end{Highlighting}
\end{Shaded}

\begin{verbatim}
##   age Medu Fedu traveltime studytime failures freetime goout Dalc Walc health
## 1  18    4    4          2         2        0        3     4    1    1      3
## 2  17    1    1          1         2        0        3     3    1    1      3
## 3  15    1    1          1         2        3        3     2    2    3      3
## 4  15    4    2          1         3        0        2     2    1    1      5
## 5  16    3    3          1         2        0        3     2    1    2      5
## 6  16    4    3          1         2        0        4     2    1    2      5
##   absences G1 G2 G3
## 1        6  5  6  6
## 2        4  5  5  6
## 3       10  7  8 10
## 4        2 15 14 15
## 5        4  6 10 10
## 6       10 15 15 15
\end{verbatim}

\hypertarget{luxe2ge-des-uxe9luxe8ves}{%
\subsubsection{L'âge des élèves}\label{luxe2ge-des-uxe9luxe8ves}}

\begin{Shaded}
\begin{Highlighting}[]
\FunctionTok{attach}\NormalTok{(data\_quanti)}
\NormalTok{data\_age}\OtherTok{=}\NormalTok{data\_quanti}

\NormalTok{data\_age}\OtherTok{=}\FunctionTok{summarise}\NormalTok{(}\FunctionTok{group\_by}\NormalTok{(data\_age,age),}\AttributeTok{n\_obs=}\FunctionTok{n}\NormalTok{()) }\CommentTok{\#on groupe par âge avec le nombre de personnes dans chaque classe}

\CommentTok{\#création du camembert}
\FunctionTok{ggplot}\NormalTok{(}\AttributeTok{data =}\NormalTok{ data\_age, }\FunctionTok{aes}\NormalTok{(}\AttributeTok{x =} \StringTok{""}\NormalTok{, }\AttributeTok{y =}\NormalTok{ n\_obs, }\AttributeTok{fill =}\NormalTok{ age)) }\SpecialCharTok{+}
  \FunctionTok{geom\_bar}\NormalTok{(}\AttributeTok{stat =} \StringTok{"identity"}\NormalTok{, }\AttributeTok{width =} \DecValTok{1}\NormalTok{) }\SpecialCharTok{+}
  \FunctionTok{coord\_polar}\NormalTok{(}\AttributeTok{theta =} \StringTok{"y"}\NormalTok{) }\SpecialCharTok{+}
  \FunctionTok{labs}\NormalTok{(}\AttributeTok{title =} \StringTok{"Répartition par âge toutes filière confondue"}\NormalTok{)}
\end{Highlighting}
\end{Shaded}

\includegraphics{projet_files/figure-latex/unnamed-chunk-16-1.pdf} La
couleur la plus claire correspond à l'âge le plus grand (22 ans), dès
que l'on passe à une couleur plus foncée, on diminue l'âge de 1. On voit
clairement ici que la majorité des étudiants ont entre 15 et 19 ans.

\begin{Shaded}
\begin{Highlighting}[]
\NormalTok{data\_age\_mat}\OtherTok{=}\NormalTok{data\_quanti\_mat}
\NormalTok{data\_age\_por}\OtherTok{=}\NormalTok{data\_quanti\_por}

\NormalTok{data\_age\_mat}\OtherTok{=}\FunctionTok{summarise}\NormalTok{(}\FunctionTok{group\_by}\NormalTok{(data\_age\_mat,age),}\AttributeTok{n\_obs\_mat=}\FunctionTok{n}\NormalTok{()) }\CommentTok{\#on groupe par âge avec le nombre de personnes dans chaque classe}
\NormalTok{data\_age\_por}\OtherTok{=}\FunctionTok{summarise}\NormalTok{(}\FunctionTok{group\_by}\NormalTok{(data\_age\_por,age),}\AttributeTok{n\_obs\_por=}\FunctionTok{n}\NormalTok{())}

\CommentTok{\#création du camembert}
\NormalTok{p1}\OtherTok{=}\FunctionTok{ggplot}\NormalTok{(}\AttributeTok{data =}\NormalTok{ data\_age\_mat, }\FunctionTok{aes}\NormalTok{(}\AttributeTok{x =} \StringTok{""}\NormalTok{, }\AttributeTok{y =}\NormalTok{ n\_obs\_mat, }\AttributeTok{fill =}\NormalTok{ age)) }\SpecialCharTok{+}
  \FunctionTok{geom\_bar}\NormalTok{(}\AttributeTok{stat =} \StringTok{"identity"}\NormalTok{, }\AttributeTok{width =} \DecValTok{1}\NormalTok{) }\SpecialCharTok{+}
  \FunctionTok{coord\_polar}\NormalTok{(}\AttributeTok{theta =} \StringTok{"y"}\NormalTok{) }\SpecialCharTok{+}
  \FunctionTok{labs}\NormalTok{(}\AttributeTok{title =} \StringTok{"Répartition par âge dans la section maths"}\NormalTok{)}

\NormalTok{p2}\OtherTok{=}\FunctionTok{ggplot}\NormalTok{(}\AttributeTok{data =}\NormalTok{ data\_age\_por, }\FunctionTok{aes}\NormalTok{(}\AttributeTok{x =} \StringTok{""}\NormalTok{, }\AttributeTok{y =}\NormalTok{ n\_obs\_por, }\AttributeTok{fill =}\NormalTok{ age)) }\SpecialCharTok{+}
  \FunctionTok{geom\_bar}\NormalTok{(}\AttributeTok{stat =} \StringTok{"identity"}\NormalTok{, }\AttributeTok{width =} \DecValTok{1}\NormalTok{) }\SpecialCharTok{+}
  \FunctionTok{coord\_polar}\NormalTok{(}\AttributeTok{theta =} \StringTok{"y"}\NormalTok{) }\SpecialCharTok{+}
  \FunctionTok{labs}\NormalTok{(}\AttributeTok{title =} \StringTok{"Répartition par âge dans la section portugais"}\NormalTok{)}

\FunctionTok{grid.arrange}\NormalTok{(p1, p2, }\AttributeTok{ncol =} \DecValTok{2}\NormalTok{)}
\end{Highlighting}
\end{Shaded}

\includegraphics{projet_files/figure-latex/unnamed-chunk-17-1.pdf} On
voit que la répartiion semble être la grosiièrement la même, en effet:

\begin{Shaded}
\begin{Highlighting}[]
\NormalTok{data\_age\_mat }\OtherTok{\textless{}{-}}\NormalTok{ data\_age\_mat }\SpecialCharTok{\%\textgreater{}\%}
  \FunctionTok{mutate}\NormalTok{(}\AttributeTok{proportion =}\NormalTok{ n\_obs\_mat }\SpecialCharTok{/} \FunctionTok{sum}\NormalTok{(n\_obs\_mat))}

\NormalTok{data\_age\_por }\OtherTok{\textless{}{-}}\NormalTok{ data\_age\_por }\SpecialCharTok{\%\textgreater{}\%}
  \FunctionTok{mutate}\NormalTok{(}\AttributeTok{proportion =}\NormalTok{ n\_obs\_por }\SpecialCharTok{/} \FunctionTok{sum}\NormalTok{(n\_obs\_por))}

\CommentTok{\#on renome de la même manière les colonnes du nombre d\textquotesingle{}étudiants pour chaque obersvation}
\NormalTok{data\_age\_mat}\SpecialCharTok{$}\NormalTok{filiere}\OtherTok{=}\FunctionTok{c}\NormalTok{(}\FunctionTok{rep}\NormalTok{(}\StringTok{"math"}\NormalTok{,}\FunctionTok{nrow}\NormalTok{(data\_age\_mat)))}
\NormalTok{data\_age\_por}\SpecialCharTok{$}\NormalTok{filiere}\OtherTok{=}\FunctionTok{c}\NormalTok{(}\FunctionTok{rep}\NormalTok{(}\StringTok{"portugais"}\NormalTok{,}\FunctionTok{nrow}\NormalTok{(data\_age\_por)))}
\FunctionTok{colnames}\NormalTok{(data\_age\_mat)[}\FunctionTok{colnames}\NormalTok{(data\_age\_mat) }\SpecialCharTok{==} \StringTok{"n\_obs\_mat"}\NormalTok{] }\OtherTok{\textless{}{-}} \StringTok{"n\_obs"}
\FunctionTok{colnames}\NormalTok{(data\_age\_por)[}\FunctionTok{colnames}\NormalTok{(data\_age\_por) }\SpecialCharTok{==} \StringTok{"n\_obs\_por"}\NormalTok{] }\OtherTok{\textless{}{-}} \StringTok{"n\_obs"}

\CommentTok{\#on concatène les deux datas frame}
\NormalTok{data\_age}\OtherTok{=}\FunctionTok{rbind}\NormalTok{(data\_age\_mat,data\_age\_por)}

\CommentTok{\#Création du graphique}

\FunctionTok{ggplot}\NormalTok{(data\_age, }\FunctionTok{aes}\NormalTok{(}\AttributeTok{x =}\NormalTok{ age, }\AttributeTok{y =}\NormalTok{ proportion, }\AttributeTok{fill =}\NormalTok{ filiere)) }\SpecialCharTok{+} 
  \FunctionTok{geom\_bar}\NormalTok{(}\AttributeTok{stat =} \StringTok{"identity"}\NormalTok{, }\AttributeTok{position =} \FunctionTok{position\_dodge}\NormalTok{()) }\SpecialCharTok{+} 
  \FunctionTok{labs}\NormalTok{(}\AttributeTok{title =} \StringTok{"Comparaison des âges dans chaque filière"}\NormalTok{, }\AttributeTok{x =}\StringTok{"âge"}\NormalTok{, }\AttributeTok{y =} \StringTok{"porportion d\textquotesingle{}étudiants au sein du groupe"}\NormalTok{) }\SpecialCharTok{+}
  \FunctionTok{scale\_fill\_manual}\NormalTok{(}\AttributeTok{values =} \FunctionTok{c}\NormalTok{(}\StringTok{"red"}\NormalTok{, }\StringTok{"blue"}\NormalTok{)) }
\end{Highlighting}
\end{Shaded}

\includegraphics{projet_files/figure-latex/unnamed-chunk-18-1.pdf} On
voit que la répartion d'âge est la même dans chaque filière

\hypertarget{quantituxe9-de-travail}{%
\subsubsection{Quantité de travail}\label{quantituxe9-de-travail}}

\begin{Shaded}
\begin{Highlighting}[]
\NormalTok{data\_stud}\OtherTok{=}\NormalTok{data\_quanti}
\NormalTok{data\_stud}\OtherTok{=}\FunctionTok{summarise}\NormalTok{(}\FunctionTok{group\_by}\NormalTok{(data\_stud,studytime),}\AttributeTok{n\_obs=}\FunctionTok{n}\NormalTok{()) }\CommentTok{\#on groupe par temps d\textquotesingle{}étude par semaine }

\NormalTok{data\_stud}\SpecialCharTok{$}\NormalTok{studytime[data\_stud}\SpecialCharTok{$}\NormalTok{studytime }\SpecialCharTok{==} \DecValTok{1}\NormalTok{] }\OtherTok{\textless{}{-}} \StringTok{"\textless{}2 hours"}
\NormalTok{data\_stud}\SpecialCharTok{$}\NormalTok{studytime[data\_stud}\SpecialCharTok{$}\NormalTok{studytime }\SpecialCharTok{==} \DecValTok{2}\NormalTok{] }\OtherTok{\textless{}{-}} \StringTok{"2 to 5 hours"}
\NormalTok{data\_stud}\SpecialCharTok{$}\NormalTok{studytime[data\_stud}\SpecialCharTok{$}\NormalTok{studytime }\SpecialCharTok{==} \DecValTok{3}\NormalTok{] }\OtherTok{\textless{}{-}} \StringTok{"5 to 10 hours"}
\NormalTok{data\_stud}\SpecialCharTok{$}\NormalTok{studytime[data\_stud}\SpecialCharTok{$}\NormalTok{studytime }\SpecialCharTok{==} \DecValTok{4}\NormalTok{] }\OtherTok{\textless{}{-}} \StringTok{"\textgreater{}10 hours"}


\FunctionTok{ggplot}\NormalTok{(data\_stud, }\FunctionTok{aes}\NormalTok{(}\AttributeTok{x =} \StringTok{""}\NormalTok{, }\AttributeTok{y =}\NormalTok{ n\_obs, }\AttributeTok{fill =} \FunctionTok{factor}\NormalTok{(studytime))) }\SpecialCharTok{+}
\FunctionTok{geom\_bar}\NormalTok{(}\AttributeTok{stat =} \StringTok{"identity"}\NormalTok{, }\AttributeTok{width =} \DecValTok{1}\NormalTok{) }\SpecialCharTok{+}
\FunctionTok{coord\_polar}\NormalTok{(}\AttributeTok{theta =} \StringTok{"y"}\NormalTok{) }\SpecialCharTok{+}
\FunctionTok{labs}\NormalTok{(}\AttributeTok{title =} \StringTok{"Répartition des temps d\textquotesingle{}étude toutes filières confondues"}\NormalTok{) }\SpecialCharTok{+}
\FunctionTok{scale\_fill\_discrete}\NormalTok{(}\AttributeTok{name =} \StringTok{"Temps d\textquotesingle{}étude"}\NormalTok{, }\AttributeTok{labels =} \FunctionTok{c}\NormalTok{(}\StringTok{"\textless{}=2 hours"}\NormalTok{, }\StringTok{"2 to 5 hours"}\NormalTok{, }\StringTok{"5 to 10 hours"}\NormalTok{, }\StringTok{"\textgreater{}10 hours"}\NormalTok{))}
\end{Highlighting}
\end{Shaded}

\includegraphics{projet_files/figure-latex/unnamed-chunk-19-1.pdf}

On voit clairement que les étudiants travaillent majoritairement moins
de 2h00 ou entre 5h00 et 10h00 par semaines.

\begin{Shaded}
\begin{Highlighting}[]
\CommentTok{\#creation data frame stud pour le groupe portugais}
\NormalTok{data\_stud\_por}\OtherTok{=}\NormalTok{data\_quanti\_por}
\NormalTok{data\_stud\_por}\OtherTok{=}\FunctionTok{summarise}\NormalTok{(}\FunctionTok{group\_by}\NormalTok{(data\_stud\_por,studytime),}\AttributeTok{n\_obs\_por=}\FunctionTok{n}\NormalTok{()) }\CommentTok{\#on groupe par temps d\textquotesingle{}étude par semaine }

\NormalTok{data\_stud\_por}\SpecialCharTok{$}\NormalTok{studytime[data\_stud\_por}\SpecialCharTok{$}\NormalTok{studytime }\SpecialCharTok{==} \DecValTok{1}\NormalTok{] }\OtherTok{\textless{}{-}} \StringTok{"\textless{}2 hours"}
\NormalTok{data\_stud\_por}\SpecialCharTok{$}\NormalTok{studytime[data\_stud\_por}\SpecialCharTok{$}\NormalTok{studytime }\SpecialCharTok{==} \DecValTok{2}\NormalTok{] }\OtherTok{\textless{}{-}} \StringTok{"2 to 5 hours"}
\NormalTok{data\_stud\_por}\SpecialCharTok{$}\NormalTok{studytime[data\_stud\_por}\SpecialCharTok{$}\NormalTok{studytime }\SpecialCharTok{==} \DecValTok{3}\NormalTok{] }\OtherTok{\textless{}{-}} \StringTok{"5 to 10 hours"}
\NormalTok{data\_stud\_por}\SpecialCharTok{$}\NormalTok{studytime[data\_stud\_por}\SpecialCharTok{$}\NormalTok{studytime }\SpecialCharTok{==} \DecValTok{4}\NormalTok{] }\OtherTok{\textless{}{-}} \StringTok{"\textgreater{}10 hours"}

\CommentTok{\#creation data frame stud pour le groupe mat b}
\NormalTok{data\_stud\_mat}\OtherTok{=}\NormalTok{data\_quanti\_mat}
\NormalTok{data\_stud\_mat}\OtherTok{=}\FunctionTok{summarise}\NormalTok{(}\FunctionTok{group\_by}\NormalTok{(data\_stud\_mat,studytime),}\AttributeTok{n\_obs\_mat=}\FunctionTok{n}\NormalTok{()) }\CommentTok{\#on groupe par temps d\textquotesingle{}étude par semaine }

\NormalTok{data\_stud\_mat}\SpecialCharTok{$}\NormalTok{studytime[data\_stud\_mat}\SpecialCharTok{$}\NormalTok{studytime }\SpecialCharTok{==} \DecValTok{1}\NormalTok{] }\OtherTok{\textless{}{-}} \StringTok{"\textless{}2 hours"}
\NormalTok{data\_stud\_mat}\SpecialCharTok{$}\NormalTok{studytime[data\_stud\_mat}\SpecialCharTok{$}\NormalTok{studytime }\SpecialCharTok{==} \DecValTok{2}\NormalTok{] }\OtherTok{\textless{}{-}} \StringTok{"2 to 5 hours"}
\NormalTok{data\_stud\_mat}\SpecialCharTok{$}\NormalTok{studytime[data\_stud\_mat}\SpecialCharTok{$}\NormalTok{studytime }\SpecialCharTok{==} \DecValTok{3}\NormalTok{] }\OtherTok{\textless{}{-}} \StringTok{"5 to 10 hours"}
\NormalTok{data\_stud\_mat}\SpecialCharTok{$}\NormalTok{studytime[data\_stud\_mat}\SpecialCharTok{$}\NormalTok{studytime }\SpecialCharTok{==} \DecValTok{4}\NormalTok{] }\OtherTok{\textless{}{-}} \StringTok{"\textgreater{}10 hours"}

\CommentTok{\#création des camemberts pour les deux sections}
\NormalTok{p1}\OtherTok{=}\FunctionTok{ggplot}\NormalTok{() }\SpecialCharTok{+}
  \CommentTok{\# Premier camembert}
  \FunctionTok{geom\_bar}\NormalTok{(}\AttributeTok{data =}\NormalTok{ data\_stud\_mat, }\FunctionTok{aes}\NormalTok{(}\AttributeTok{x =} \StringTok{""}\NormalTok{, }\AttributeTok{y =}\NormalTok{ n\_obs\_mat, }\AttributeTok{fill =} \FunctionTok{factor}\NormalTok{(studytime)), }\AttributeTok{stat =} \StringTok{"identity"}\NormalTok{, }\AttributeTok{width =} \DecValTok{1}\NormalTok{) }\SpecialCharTok{+}
  \FunctionTok{coord\_polar}\NormalTok{(}\AttributeTok{theta =} \StringTok{"y"}\NormalTok{) }\SpecialCharTok{+}
  \FunctionTok{theme\_void}\NormalTok{() }\SpecialCharTok{+}
  \FunctionTok{labs}\NormalTok{(}\AttributeTok{title =} \StringTok{"Temps d\textquotesingle{}étude par semaine dans la section maths (à gauche) et portugaise (à droite)"}\NormalTok{) }

  \CommentTok{\# Deuxième camembert}
\NormalTok{p2}\OtherTok{=}\FunctionTok{ggplot}\NormalTok{() }\SpecialCharTok{+}
  \FunctionTok{geom\_bar}\NormalTok{(}\AttributeTok{data =}\NormalTok{ data\_stud\_por, }\FunctionTok{aes}\NormalTok{(}\AttributeTok{x =} \StringTok{""}\NormalTok{, }\AttributeTok{y =}\NormalTok{ n\_obs\_por, }\AttributeTok{fill =} \FunctionTok{factor}\NormalTok{(studytime)), }\AttributeTok{stat =} \StringTok{"identity"}\NormalTok{, }\AttributeTok{width =} \DecValTok{1}\NormalTok{) }\SpecialCharTok{+}
  \FunctionTok{coord\_polar}\NormalTok{(}\AttributeTok{theta =} \StringTok{"y"}\NormalTok{) }\SpecialCharTok{+}
  \FunctionTok{theme\_void}\NormalTok{() }

\FunctionTok{grid.arrange}\NormalTok{(p1, p2, }\AttributeTok{ncol =} \DecValTok{2}\NormalTok{)}
\end{Highlighting}
\end{Shaded}

\includegraphics{projet_files/figure-latex/unnamed-chunk-20-1.pdf}

\begin{Shaded}
\begin{Highlighting}[]
\NormalTok{data\_stud\_mat}
\end{Highlighting}
\end{Shaded}

\begin{verbatim}
## # A tibble: 4 x 2
##   studytime     n_obs_mat
##   <chr>             <int>
## 1 <2 hours            105
## 2 2 to 5 hours        198
## 3 5 to 10 hours        65
## 4 >10 hours            27
\end{verbatim}

On voit qu'il y a plus de personnes qui travaillent moins de deux heures
par semaine dans la section portiguaise tandis qu'il y a moins de
personnes qui travaillent plus de 10h00 dans cette même section. Le
nombre d'étudiants travaillant entre 5 et 10 heures semble être a peu
près le même. En effet:

\begin{Shaded}
\begin{Highlighting}[]
\CommentTok{\#on calcul la porportion pour pouvoir comparer}
\NormalTok{data\_stud\_mat }\OtherTok{\textless{}{-}}\NormalTok{ data\_stud\_mat }\SpecialCharTok{\%\textgreater{}\%}
  \FunctionTok{mutate}\NormalTok{(}\AttributeTok{proportion =}\NormalTok{ n\_obs\_mat }\SpecialCharTok{/} \FunctionTok{sum}\NormalTok{(n\_obs\_mat))}

\NormalTok{data\_stud\_por }\OtherTok{\textless{}{-}}\NormalTok{ data\_stud\_por }\SpecialCharTok{\%\textgreater{}\%}
  \FunctionTok{mutate}\NormalTok{(}\AttributeTok{proportion =}\NormalTok{ n\_obs\_por }\SpecialCharTok{/} \FunctionTok{sum}\NormalTok{(n\_obs\_por))}

\CommentTok{\#on renome de la même manière les colonnes du nombre d\textquotesingle{}étudiants pour chaque obersvation}
\NormalTok{data\_stud\_mat}\SpecialCharTok{$}\NormalTok{filiere}\OtherTok{=}\FunctionTok{c}\NormalTok{(}\FunctionTok{rep}\NormalTok{(}\StringTok{"math"}\NormalTok{,}\FunctionTok{nrow}\NormalTok{(data\_stud\_mat)))}
\NormalTok{data\_stud\_por}\SpecialCharTok{$}\NormalTok{filiere}\OtherTok{=}\FunctionTok{c}\NormalTok{(}\FunctionTok{rep}\NormalTok{(}\StringTok{"portugais"}\NormalTok{,}\FunctionTok{nrow}\NormalTok{(data\_stud\_por)))}
\FunctionTok{colnames}\NormalTok{(data\_stud\_mat)[}\FunctionTok{colnames}\NormalTok{(data\_stud\_mat) }\SpecialCharTok{==} \StringTok{"n\_obs\_mat"}\NormalTok{] }\OtherTok{\textless{}{-}} \StringTok{"n\_obs"}
\FunctionTok{colnames}\NormalTok{(data\_stud\_por)[}\FunctionTok{colnames}\NormalTok{(data\_stud\_por) }\SpecialCharTok{==} \StringTok{"n\_obs\_por"}\NormalTok{] }\OtherTok{\textless{}{-}} \StringTok{"n\_obs"}

\CommentTok{\#on concatène les deux datas frame}
\NormalTok{data\_stud}\OtherTok{=}\FunctionTok{rbind}\NormalTok{(data\_stud\_mat,data\_stud\_por)}

\CommentTok{\#Création du graphique}

\FunctionTok{ggplot}\NormalTok{(data\_stud, }\FunctionTok{aes}\NormalTok{(}\AttributeTok{x =}\NormalTok{ studytime, }\AttributeTok{y =}\NormalTok{ proportion, }\AttributeTok{fill =}\NormalTok{ filiere)) }\SpecialCharTok{+} 
  \FunctionTok{geom\_bar}\NormalTok{(}\AttributeTok{stat =} \StringTok{"identity"}\NormalTok{, }\AttributeTok{position =} \FunctionTok{position\_dodge}\NormalTok{()) }\SpecialCharTok{+} 
  \FunctionTok{labs}\NormalTok{(}\AttributeTok{title =} \StringTok{"Comparaison du temps de travail entre deux filières"}\NormalTok{, }\AttributeTok{x =} \StringTok{"Temps de travail par semaine"}\NormalTok{, }\AttributeTok{y =} \StringTok{"Proportion d\textquotesingle{}étudiants au sein de chaque groupe"}\NormalTok{) }\SpecialCharTok{+}
  \FunctionTok{scale\_fill\_manual}\NormalTok{(}\AttributeTok{values =} \FunctionTok{c}\NormalTok{(}\StringTok{"red"}\NormalTok{, }\StringTok{"blue"}\NormalTok{))}
\end{Highlighting}
\end{Shaded}

\includegraphics{projet_files/figure-latex/unnamed-chunk-21-1.pdf}

On s'aperçoit donc que les élèves dans la filière mathématiques
travaillent plus

\begin{Shaded}
\begin{Highlighting}[]
\NormalTok{data\_quanti}\SpecialCharTok{$}\NormalTok{studytime[data\_quanti}\SpecialCharTok{$}\NormalTok{studytime }\SpecialCharTok{==} \DecValTok{1}\NormalTok{] }\OtherTok{\textless{}{-}} \StringTok{"\textless{}2 hours"}
\NormalTok{data\_quanti}\SpecialCharTok{$}\NormalTok{studytime[data\_quanti}\SpecialCharTok{$}\NormalTok{studytime }\SpecialCharTok{==} \DecValTok{2}\NormalTok{] }\OtherTok{\textless{}{-}} \StringTok{"2 to 5 hours"}
\NormalTok{data\_quanti}\SpecialCharTok{$}\NormalTok{studytime[data\_quanti}\SpecialCharTok{$}\NormalTok{studytime }\SpecialCharTok{==} \DecValTok{3}\NormalTok{] }\OtherTok{\textless{}{-}} \StringTok{"5 to 10 hours"}
\NormalTok{data\_quanti}\SpecialCharTok{$}\NormalTok{studytime[data\_quanti}\SpecialCharTok{$}\NormalTok{studytime }\SpecialCharTok{==} \DecValTok{4}\NormalTok{] }\OtherTok{\textless{}{-}} \StringTok{"\textgreater{}10 hours"}

\FunctionTok{ggplot}\NormalTok{(data\_quanti, }\FunctionTok{aes}\NormalTok{(}\AttributeTok{x =}\NormalTok{ G3, }\AttributeTok{y =}\NormalTok{ studytime)) }\SpecialCharTok{+}
  \FunctionTok{geom\_boxplot}\NormalTok{() }
\end{Highlighting}
\end{Shaded}

\includegraphics{projet_files/figure-latex/unnamed-chunk-22-1.pdf}

On voit que globalement, les élèves qui travaillent plus ont de
meilleures notes (comportement bizarre à vérifier)

\hypertarget{absences-des-uxe9tudiants}{%
\subsubsection{Absences des étudiants}\label{absences-des-uxe9tudiants}}

\begin{Shaded}
\begin{Highlighting}[]
\FunctionTok{ggplot}\NormalTok{(df[df}\SpecialCharTok{$}\NormalTok{address }\SpecialCharTok{==} \StringTok{\textquotesingle{}U\textquotesingle{}}\NormalTok{,], }\FunctionTok{aes}\NormalTok{(}\AttributeTok{x=}\NormalTok{absences)) }\SpecialCharTok{+}
  \FunctionTok{geom\_histogram}\NormalTok{(}\FunctionTok{aes}\NormalTok{(}\AttributeTok{y =}\NormalTok{ ..count.. }\SpecialCharTok{/} \FunctionTok{sum}\NormalTok{(..count..)), }\AttributeTok{binwidth=}\DecValTok{1}\NormalTok{, }\AttributeTok{fill=}\StringTok{"steelblue"}\NormalTok{, }\AttributeTok{color=}\StringTok{"white"}\NormalTok{) }\SpecialCharTok{+}
  \FunctionTok{labs}\NormalTok{(}\AttributeTok{title=}\StringTok{"Distribution des absences des étudiants vivants en ville"}\NormalTok{,}
       \AttributeTok{x=}\StringTok{"Nombre d\textquotesingle{}absences"}\NormalTok{, }\AttributeTok{y=}\StringTok{"Proportion d\textquotesingle{}étudiants"}\NormalTok{) }\SpecialCharTok{+}
  \FunctionTok{theme\_minimal}\NormalTok{()}
\end{Highlighting}
\end{Shaded}

\begin{verbatim}
## Warning: The dot-dot notation (`..count..`) was deprecated in ggplot2 3.4.0.
## i Please use `after_stat(count)` instead.
\end{verbatim}

\includegraphics{projet_files/figure-latex/unnamed-chunk-23-1.pdf}

\hypertarget{variables-qualitatives-uxe0-modalituxe9s-numuxe9riques}{%
\subsection{Variables qualitatives à modalités
numériques}\label{variables-qualitatives-uxe0-modalituxe9s-numuxe9riques}}

\hypertarget{santuxe9-des-uxe9tudiants}{%
\subsubsection{Santé des étudiants}\label{santuxe9-des-uxe9tudiants}}

\begin{Shaded}
\begin{Highlighting}[]
\NormalTok{data\_health}\OtherTok{=}\NormalTok{data\_quanti}
\NormalTok{data\_health}\OtherTok{=}\FunctionTok{summarise}\NormalTok{(}\FunctionTok{group\_by}\NormalTok{(data\_health,health),}\AttributeTok{n\_obs=}\FunctionTok{n}\NormalTok{())}


\FunctionTok{ggplot}\NormalTok{(data\_health, }\FunctionTok{aes}\NormalTok{(}\AttributeTok{x =} \StringTok{""}\NormalTok{, }\AttributeTok{y =}\NormalTok{ n\_obs, }\AttributeTok{fill =} \FunctionTok{factor}\NormalTok{(health))) }\SpecialCharTok{+}
  \FunctionTok{geom\_bar}\NormalTok{(}\AttributeTok{stat =} \StringTok{"identity"}\NormalTok{, }\AttributeTok{width =} \DecValTok{1}\NormalTok{) }\SpecialCharTok{+}
  \FunctionTok{coord\_polar}\NormalTok{(}\AttributeTok{theta =} \StringTok{"y"}\NormalTok{) }\SpecialCharTok{+}
  \FunctionTok{labs}\NormalTok{(}\AttributeTok{title =} \StringTok{"Santé des étudiants"}\NormalTok{) }\SpecialCharTok{+}
  \FunctionTok{scale\_fill\_discrete}\NormalTok{(}\AttributeTok{name =} \StringTok{"Niveau de santé"}\NormalTok{, }\AttributeTok{labels =} \FunctionTok{c}\NormalTok{(}\DecValTok{1}\NormalTok{,}\DecValTok{2}\NormalTok{,}\DecValTok{3}\NormalTok{,}\DecValTok{4}\NormalTok{,}\DecValTok{5}\NormalTok{))}
\end{Highlighting}
\end{Shaded}

\includegraphics{projet_files/figure-latex/unnamed-chunk-24-1.pdf} On
voit que la plupart des étudiant sont en bonne santé

\begin{Shaded}
\begin{Highlighting}[]
\NormalTok{data\_quanti}\SpecialCharTok{$}\NormalTok{health}\OtherTok{=}\FunctionTok{factor}\NormalTok{(data\_quanti}\SpecialCharTok{$}\NormalTok{health)}
\FunctionTok{ggplot}\NormalTok{(data\_quanti, }\FunctionTok{aes}\NormalTok{(}\AttributeTok{x =}\NormalTok{ G3, }\AttributeTok{y =}\NormalTok{ health)) }\SpecialCharTok{+}
  \FunctionTok{geom\_boxplot}\NormalTok{() }
\end{Highlighting}
\end{Shaded}

\includegraphics{projet_files/figure-latex/unnamed-chunk-25-1.pdf}

On voit que les étudiants en meilleure santé ont une meilleure réussite

\begin{Shaded}
\begin{Highlighting}[]
\FunctionTok{ggplot}\NormalTok{(df, }\FunctionTok{aes}\NormalTok{(}\AttributeTok{x =}\NormalTok{ age, }\AttributeTok{y =}\NormalTok{ G3, }\AttributeTok{color =}\NormalTok{ paid)) }\SpecialCharTok{+}
  \FunctionTok{geom\_point}\NormalTok{(}\AttributeTok{alpha =} \FloatTok{0.7}\NormalTok{) }\SpecialCharTok{+}
  \FunctionTok{facet\_wrap}\NormalTok{(}\SpecialCharTok{\textasciitilde{}}\NormalTok{paid) }\SpecialCharTok{+}
  \FunctionTok{labs}\NormalTok{(}\AttributeTok{title =} \StringTok{"Distribution de l\textquotesingle{}âge et de la note finale en fonction cours particuliers et de l\textquotesingle{}âge"}\NormalTok{,}
       \AttributeTok{x =} \StringTok{"Âge"}\NormalTok{, }\AttributeTok{y =} \StringTok{"Note finale"}\NormalTok{)}
\end{Highlighting}
\end{Shaded}

\includegraphics{projet_files/figure-latex/unnamed-chunk-26-1.pdf}
Curieusement, les résultats semblent meilleur pour ceux qui n'on pas
pris de cours

\hypertarget{machine-learning-classification-de-la-ruxe9ussite-scolaire}{%
\section{4. Machine Learning : Classification de la réussite
scolaire}\label{machine-learning-classification-de-la-ruxe9ussite-scolaire}}

Dans cette partie, nous nous concentrons sur la mise en place de
méthodes de classification afin de prédire la variable RS (réussite
scolaire). Nous nous intéresserons essentiellement à la comparaison des
résultats de chacune des méthodes. Les méthodes utilisées seront
évaluées avec leur accuracy et leur courbe ROC.

\hypertarget{a-suxe9paration-du-jeu-de-donnuxe9es}{%
\subsection{a) Séparation du jeu de
données}\label{a-suxe9paration-du-jeu-de-donnuxe9es}}

Ici, nous découpons notre dataset en jeu d'entraînement et jeu de test.
Le ratio utilisé est \(\frac{1}{5}\) pour le jeu de test. Tout d'abord
on modifie notre jeu de données pour le préparer pour la classification
en retirant les notes.

\begin{Shaded}
\begin{Highlighting}[]
\CommentTok{\# Suppresion des colonnes}
\NormalTok{X }\OtherTok{=} \FunctionTok{subset}\NormalTok{(df, }\AttributeTok{select =} \SpecialCharTok{{-}}\FunctionTok{c}\NormalTok{(G1,G2,G3,Moy) )}

\FunctionTok{set.seed}\NormalTok{(}\DecValTok{1}\NormalTok{)}
\NormalTok{n }\OtherTok{\textless{}{-}} \FunctionTok{nrow}\NormalTok{(X)}
\NormalTok{p }\OtherTok{\textless{}{-}} \FunctionTok{ncol}\NormalTok{(X)}\SpecialCharTok{{-}}\DecValTok{1}
\NormalTok{test.ratio }\OtherTok{\textless{}{-}}\NormalTok{ .}\DecValTok{2} \CommentTok{\# ratio of test/train samples}
\NormalTok{n.test }\OtherTok{\textless{}{-}} \FunctionTok{round}\NormalTok{(n}\SpecialCharTok{*}\NormalTok{test.ratio)}
\NormalTok{n.test}
\end{Highlighting}
\end{Shaded}

\begin{verbatim}
## [1] 209
\end{verbatim}

\begin{Shaded}
\begin{Highlighting}[]
\NormalTok{tr }\OtherTok{\textless{}{-}} \FunctionTok{sample}\NormalTok{(}\DecValTok{1}\SpecialCharTok{:}\NormalTok{n,n.test)}
\NormalTok{df.test }\OtherTok{\textless{}{-}}\NormalTok{ X[tr,]}
\NormalTok{df.train }\OtherTok{\textless{}{-}}\NormalTok{ X[}\SpecialCharTok{{-}}\NormalTok{tr,]}
\end{Highlighting}
\end{Shaded}

\hypertarget{b-lda}{%
\subsection{b) LDA}\label{b-lda}}

\begin{Shaded}
\begin{Highlighting}[]
\NormalTok{res\_lda}\OtherTok{=}\FunctionTok{lda}\NormalTok{(df.train}\SpecialCharTok{$}\NormalTok{RS }\SpecialCharTok{\textasciitilde{}}\NormalTok{., }\AttributeTok{data=}\NormalTok{df.train)}
\NormalTok{pred\_lda }\OtherTok{\textless{}{-}} \FunctionTok{predict}\NormalTok{(res\_lda,}\AttributeTok{newdata=}\NormalTok{df.test)}\SpecialCharTok{$}\NormalTok{posterior[,}\DecValTok{2}\NormalTok{] }

\CommentTok{\# Table de confusion}
\FunctionTok{table}\NormalTok{(df.test}\SpecialCharTok{$}\NormalTok{RS,}\FunctionTok{predict}\NormalTok{(res\_lda,}\AttributeTok{newdata=}\NormalTok{df.test)}\SpecialCharTok{$}\NormalTok{class)}
\end{Highlighting}
\end{Shaded}

\begin{verbatim}
##               
##                admis exclusion redoublement
##   admis          137         1            5
##   exclusion       17        14            2
##   redoublement    23         4            6
\end{verbatim}

\begin{Shaded}
\begin{Highlighting}[]
\CommentTok{\# Courbe ROC}
\NormalTok{ROC\_lda }\OtherTok{\textless{}{-}} \FunctionTok{roc}\NormalTok{(df.test}\SpecialCharTok{$}\NormalTok{RS, pred\_lda)}
\end{Highlighting}
\end{Shaded}

\begin{verbatim}
## Warning in roc.default(df.test$RS, pred_lda): 'response' has more than two
## levels. Consider setting 'levels' explicitly or using 'multiclass.roc' instead
\end{verbatim}

\begin{verbatim}
## Setting levels: control = admis, case = exclusion
\end{verbatim}

\begin{verbatim}
## Setting direction: controls < cases
\end{verbatim}

\begin{Shaded}
\begin{Highlighting}[]
\FunctionTok{plot}\NormalTok{(ROC\_lda, }\AttributeTok{print.auc=}\ConstantTok{TRUE}\NormalTok{,  }\AttributeTok{print.auc.y =} \FloatTok{0.5}\NormalTok{)}
\end{Highlighting}
\end{Shaded}

\includegraphics{projet_files/figure-latex/unnamed-chunk-28-1.pdf}

\begin{Shaded}
\begin{Highlighting}[]
\NormalTok{ROC\_lda}\SpecialCharTok{$}\NormalTok{auc}
\end{Highlighting}
\end{Shaded}

\begin{verbatim}
## Area under the curve: 0.8226
\end{verbatim}

\begin{Shaded}
\begin{Highlighting}[]
\CommentTok{\# Accuracy }
\NormalTok{accuracy\_lda }\OtherTok{=} \FunctionTok{mean}\NormalTok{(df.test}\SpecialCharTok{$}\NormalTok{RS}\SpecialCharTok{==}\FunctionTok{predict}\NormalTok{(res\_lda,}\AttributeTok{newdata=}\NormalTok{df.test)}\SpecialCharTok{$}\NormalTok{class)}
\FunctionTok{print}\NormalTok{(}\StringTok{"accuracy lda = "}\NormalTok{)}
\end{Highlighting}
\end{Shaded}

\begin{verbatim}
## [1] "accuracy lda = "
\end{verbatim}

\begin{Shaded}
\begin{Highlighting}[]
\FunctionTok{print}\NormalTok{(accuracy\_lda)}
\end{Highlighting}
\end{Shaded}

\begin{verbatim}
## [1] 0.7511962
\end{verbatim}

\hypertarget{c-qda}{%
\subsection{c) QDA}\label{c-qda}}

\begin{Shaded}
\begin{Highlighting}[]
\NormalTok{res\_qda }\OtherTok{=} \FunctionTok{qda}\NormalTok{(df.train}\SpecialCharTok{$}\NormalTok{RS}\SpecialCharTok{\textasciitilde{}}\NormalTok{., }\AttributeTok{data=}\NormalTok{df.train)}
\NormalTok{pred\_qda }\OtherTok{\textless{}{-}} \FunctionTok{predict}\NormalTok{(res\_qda,}\AttributeTok{newdata=}\NormalTok{df.test)}\SpecialCharTok{$}\NormalTok{posterior[,}\DecValTok{2}\NormalTok{] }

\CommentTok{\# Table de confusion}
\FunctionTok{table}\NormalTok{(df.test}\SpecialCharTok{$}\NormalTok{RS,}\FunctionTok{predict}\NormalTok{(res\_qda,}\AttributeTok{newdata=}\NormalTok{df.test)}\SpecialCharTok{$}\NormalTok{class)}
\end{Highlighting}
\end{Shaded}

\begin{verbatim}
##               
##                admis exclusion redoublement
##   admis          128        12            3
##   exclusion       12        17            4
##   redoublement    19         7            7
\end{verbatim}

\begin{Shaded}
\begin{Highlighting}[]
\CommentTok{\# Courbe ROC}
\NormalTok{ROC\_qda }\OtherTok{\textless{}{-}} \FunctionTok{roc}\NormalTok{(df.test}\SpecialCharTok{$}\NormalTok{RS, pred\_qda)}
\end{Highlighting}
\end{Shaded}

\begin{verbatim}
## Warning in roc.default(df.test$RS, pred_qda): 'response' has more than two
## levels. Consider setting 'levels' explicitly or using 'multiclass.roc' instead
\end{verbatim}

\begin{verbatim}
## Setting levels: control = admis, case = exclusion
\end{verbatim}

\begin{verbatim}
## Setting direction: controls < cases
\end{verbatim}

\begin{Shaded}
\begin{Highlighting}[]
\FunctionTok{plot}\NormalTok{(ROC\_qda, }\AttributeTok{print.auc=}\ConstantTok{TRUE}\NormalTok{,  }\AttributeTok{print.auc.y =} \FloatTok{0.5}\NormalTok{)}
\end{Highlighting}
\end{Shaded}

\includegraphics{projet_files/figure-latex/unnamed-chunk-29-1.pdf}

\begin{Shaded}
\begin{Highlighting}[]
\NormalTok{ROC\_qda}\SpecialCharTok{$}\NormalTok{auc}
\end{Highlighting}
\end{Shaded}

\begin{verbatim}
## Area under the curve: 0.6944
\end{verbatim}

\begin{Shaded}
\begin{Highlighting}[]
\CommentTok{\# Accuracy }
\NormalTok{accuracy\_qda }\OtherTok{=} \FunctionTok{mean}\NormalTok{(df.test}\SpecialCharTok{$}\NormalTok{RS}\SpecialCharTok{==}\FunctionTok{predict}\NormalTok{(res\_qda,}\AttributeTok{newdata=}\NormalTok{df.test)}\SpecialCharTok{$}\NormalTok{class)}
\FunctionTok{print}\NormalTok{(}\StringTok{"accuracy qda = "}\NormalTok{)}
\end{Highlighting}
\end{Shaded}

\begin{verbatim}
## [1] "accuracy qda = "
\end{verbatim}

\begin{Shaded}
\begin{Highlighting}[]
\FunctionTok{print}\NormalTok{(accuracy\_qda)}
\end{Highlighting}
\end{Shaded}

\begin{verbatim}
## [1] 0.7272727
\end{verbatim}

\hypertarget{d-stepwise}{%
\subsection{d) Stepwise}\label{d-stepwise}}

\begin{Shaded}
\begin{Highlighting}[]
\NormalTok{stepwise\_lda}\OtherTok{=}\FunctionTok{stepclass}\NormalTok{(RS}\SpecialCharTok{\textasciitilde{}}\NormalTok{., }\AttributeTok{data=}\NormalTok{df.train, }\AttributeTok{method=}\StringTok{"lda"}\NormalTok{, }\AttributeTok{direction=}\StringTok{"backward"}\NormalTok{)}
\end{Highlighting}
\end{Shaded}

\begin{verbatim}
##  `stepwise classification', using 10-fold cross-validated correctness rate of method lda'.
\end{verbatim}

\begin{verbatim}
## 835 observations of 30 variables in 3 classes; direction: backward
\end{verbatim}

\begin{verbatim}
## stop criterion: improvement less than 5%.
\end{verbatim}

\begin{verbatim}
## Warning in cv.rate(vars = start.vars, data = data, grouping = grouping, :
## error(s) in modeling/prediction step
\end{verbatim}

\begin{verbatim}
## correctness rate: 0;  starting variables (30): school, sex, age, address, famsize, Pstatus, Medu, Fedu, Mjob, Fjob, reason, guardian, traveltime, studytime, failures, schoolsup, famsup, paid, activities, nursery, higher, internet, romantic, famrel, freetime, goout, Dalc, Walc, health, absences
\end{verbatim}

\begin{verbatim}
## Warning in cv.rate(trymodel, data = data, grouping = grouping, method = method,
## : error(s) in modeling/prediction step
\end{verbatim}

\begin{verbatim}
## Warning in cv.rate(trymodel, data = data, grouping = grouping, method = method,
## : error(s) in modeling/prediction step

## Warning in cv.rate(trymodel, data = data, grouping = grouping, method = method,
## : error(s) in modeling/prediction step

## Warning in cv.rate(trymodel, data = data, grouping = grouping, method = method,
## : error(s) in modeling/prediction step

## Warning in cv.rate(trymodel, data = data, grouping = grouping, method = method,
## : error(s) in modeling/prediction step

## Warning in cv.rate(trymodel, data = data, grouping = grouping, method = method,
## : error(s) in modeling/prediction step

## Warning in cv.rate(trymodel, data = data, grouping = grouping, method = method,
## : error(s) in modeling/prediction step

## Warning in cv.rate(trymodel, data = data, grouping = grouping, method = method,
## : error(s) in modeling/prediction step

## Warning in cv.rate(trymodel, data = data, grouping = grouping, method = method,
## : error(s) in modeling/prediction step

## Warning in cv.rate(trymodel, data = data, grouping = grouping, method = method,
## : error(s) in modeling/prediction step

## Warning in cv.rate(trymodel, data = data, grouping = grouping, method = method,
## : error(s) in modeling/prediction step

## Warning in cv.rate(trymodel, data = data, grouping = grouping, method = method,
## : error(s) in modeling/prediction step

## Warning in cv.rate(trymodel, data = data, grouping = grouping, method = method,
## : error(s) in modeling/prediction step

## Warning in cv.rate(trymodel, data = data, grouping = grouping, method = method,
## : error(s) in modeling/prediction step

## Warning in cv.rate(trymodel, data = data, grouping = grouping, method = method,
## : error(s) in modeling/prediction step

## Warning in cv.rate(trymodel, data = data, grouping = grouping, method = method,
## : error(s) in modeling/prediction step

## Warning in cv.rate(trymodel, data = data, grouping = grouping, method = method,
## : error(s) in modeling/prediction step

## Warning in cv.rate(trymodel, data = data, grouping = grouping, method = method,
## : error(s) in modeling/prediction step

## Warning in cv.rate(trymodel, data = data, grouping = grouping, method = method,
## : error(s) in modeling/prediction step

## Warning in cv.rate(trymodel, data = data, grouping = grouping, method = method,
## : error(s) in modeling/prediction step

## Warning in cv.rate(trymodel, data = data, grouping = grouping, method = method,
## : error(s) in modeling/prediction step

## Warning in cv.rate(trymodel, data = data, grouping = grouping, method = method,
## : error(s) in modeling/prediction step

## Warning in cv.rate(trymodel, data = data, grouping = grouping, method = method,
## : error(s) in modeling/prediction step

## Warning in cv.rate(trymodel, data = data, grouping = grouping, method = method,
## : error(s) in modeling/prediction step

## Warning in cv.rate(trymodel, data = data, grouping = grouping, method = method,
## : error(s) in modeling/prediction step

## Warning in cv.rate(trymodel, data = data, grouping = grouping, method = method,
## : error(s) in modeling/prediction step

## Warning in cv.rate(trymodel, data = data, grouping = grouping, method = method,
## : error(s) in modeling/prediction step

## Warning in cv.rate(trymodel, data = data, grouping = grouping, method = method,
## : error(s) in modeling/prediction step

## Warning in cv.rate(trymodel, data = data, grouping = grouping, method = method,
## : error(s) in modeling/prediction step

## Warning in cv.rate(trymodel, data = data, grouping = grouping, method = method,
## : error(s) in modeling/prediction step
\end{verbatim}

\begin{verbatim}
## 
##  hr.elapsed min.elapsed sec.elapsed 
##       0.000       0.000       2.786
\end{verbatim}

\begin{Shaded}
\begin{Highlighting}[]
\NormalTok{stepwise\_lda}
\end{Highlighting}
\end{Shaded}

\begin{verbatim}
## method      : lda 
## final model : RS ~ school + sex + age + address + famsize + Pstatus + Medu + 
##     Fedu + Mjob + Fjob + reason + guardian + traveltime + studytime + 
##     failures + schoolsup + famsup + paid + activities + nursery + 
##     higher + internet + romantic + famrel + freetime + goout + 
##     Dalc + Walc + health + absences
## <environment: 0x563fb1c0f6c0>
## 
## correctness rate = 0
\end{verbatim}

\begin{Shaded}
\begin{Highlighting}[]
\NormalTok{res\_stepwise\_lda }\OtherTok{=} \FunctionTok{lda}\NormalTok{(stepwise\_lda}\SpecialCharTok{$}\NormalTok{formula, }\AttributeTok{data=}\NormalTok{df.train)}

\NormalTok{pred\_lda\_step }\OtherTok{\textless{}{-}} \FunctionTok{predict}\NormalTok{(res\_stepwise\_lda,}\AttributeTok{newdata=}\NormalTok{df.test)}\SpecialCharTok{$}\NormalTok{posterior[,}\DecValTok{2}\NormalTok{] }

\CommentTok{\# Table de confusion}
\FunctionTok{table}\NormalTok{(df.test}\SpecialCharTok{$}\NormalTok{RS, }\FunctionTok{predict}\NormalTok{(res\_stepwise\_lda,}\AttributeTok{newdata=}\NormalTok{df.test)}\SpecialCharTok{$}\NormalTok{class)}
\end{Highlighting}
\end{Shaded}

\begin{verbatim}
##               
##                admis exclusion redoublement
##   admis          137         1            5
##   exclusion       17        14            2
##   redoublement    23         4            6
\end{verbatim}

\begin{Shaded}
\begin{Highlighting}[]
\CommentTok{\# Courbe ROC}
\NormalTok{ROC\_lda\_step }\OtherTok{\textless{}{-}} \FunctionTok{roc}\NormalTok{(df.test}\SpecialCharTok{$}\NormalTok{RS, pred\_lda)}
\end{Highlighting}
\end{Shaded}

\begin{verbatim}
## Warning in roc.default(df.test$RS, pred_lda): 'response' has more than two
## levels. Consider setting 'levels' explicitly or using 'multiclass.roc' instead
\end{verbatim}

\begin{verbatim}
## Setting levels: control = admis, case = exclusion
\end{verbatim}

\begin{verbatim}
## Setting direction: controls < cases
\end{verbatim}

\begin{Shaded}
\begin{Highlighting}[]
\FunctionTok{plot}\NormalTok{(ROC\_lda\_step, }\AttributeTok{print.auc=}\ConstantTok{TRUE}\NormalTok{,  }\AttributeTok{print.auc.y =} \FloatTok{0.5}\NormalTok{)}
\end{Highlighting}
\end{Shaded}

\includegraphics{projet_files/figure-latex/unnamed-chunk-30-1.pdf}

\begin{Shaded}
\begin{Highlighting}[]
\NormalTok{ROC\_lda\_step}\SpecialCharTok{$}\NormalTok{auc}
\end{Highlighting}
\end{Shaded}

\begin{verbatim}
## Area under the curve: 0.8226
\end{verbatim}

\begin{Shaded}
\begin{Highlighting}[]
\CommentTok{\# Accuracy }
\NormalTok{accuracy\_lda\_stepwise }\OtherTok{=} \FunctionTok{mean}\NormalTok{(df.test}\SpecialCharTok{$}\NormalTok{RS}\SpecialCharTok{==} \FunctionTok{predict}\NormalTok{(res\_stepwise\_lda,}\AttributeTok{newdata=}\NormalTok{df.test)}\SpecialCharTok{$}\NormalTok{class)}
\FunctionTok{print}\NormalTok{(}\StringTok{"accuracy lda stepwise = "}\NormalTok{)}
\end{Highlighting}
\end{Shaded}

\begin{verbatim}
## [1] "accuracy lda stepwise = "
\end{verbatim}

\begin{Shaded}
\begin{Highlighting}[]
\FunctionTok{print}\NormalTok{(accuracy\_lda\_stepwise)}
\end{Highlighting}
\end{Shaded}

\begin{verbatim}
## [1] 0.7511962
\end{verbatim}

\hypertarget{e-random-forest}{%
\subsection{e) Random Forest}\label{e-random-forest}}

\begin{Shaded}
\begin{Highlighting}[]
\NormalTok{res\_RF }\OtherTok{\textless{}{-}} \FunctionTok{randomForest}\NormalTok{(RS}\SpecialCharTok{\textasciitilde{}}\NormalTok{.,df.train)}
\NormalTok{res\_RF}
\end{Highlighting}
\end{Shaded}

\begin{verbatim}
## 
## Call:
##  randomForest(formula = RS ~ ., data = df.train) 
##                Type of random forest: classification
##                      Number of trees: 500
## No. of variables tried at each split: 5
## 
##         OOB estimate of  error rate: 31.98%
## Confusion matrix:
##              admis exclusion redoublement class.error
## admis          526        34           20  0.09310345
## exclusion       95        34           11  0.75714286
## redoublement    87        20            8  0.93043478
\end{verbatim}

\begin{Shaded}
\begin{Highlighting}[]
\FunctionTok{plot}\NormalTok{(res\_RF)}
\end{Highlighting}
\end{Shaded}

\includegraphics{projet_files/figure-latex/unnamed-chunk-31-1.pdf}

\begin{Shaded}
\begin{Highlighting}[]
\DocumentationTok{\#\# prédiction :}
\NormalTok{pred\_RF }\OtherTok{\textless{}{-}} \FunctionTok{predict}\NormalTok{(res\_RF,}\AttributeTok{newdata=}\NormalTok{df.test)}

\DocumentationTok{\#\# Table confusion et accuracy :}
\FunctionTok{table}\NormalTok{(df.test}\SpecialCharTok{$}\NormalTok{RS, }\FunctionTok{predict}\NormalTok{(res\_RF,}\AttributeTok{newdata=}\NormalTok{df.test,}\AttributeTok{type=}\StringTok{"class"}\NormalTok{))}
\end{Highlighting}
\end{Shaded}

\begin{verbatim}
##               
##                admis exclusion redoublement
##   admis          135         7            1
##   exclusion       16        14            3
##   redoublement    26         6            1
\end{verbatim}

\begin{Shaded}
\begin{Highlighting}[]
\DocumentationTok{\#\# aire sous courbe ROC}
\NormalTok{pred\_RF }\OtherTok{=} \FunctionTok{predict}\NormalTok{(res\_RF, df.test, }\AttributeTok{type=}\StringTok{"prob"}\NormalTok{)[,}\DecValTok{2}\NormalTok{] }
\NormalTok{ROC\_RF }\OtherTok{\textless{}{-}} \FunctionTok{roc}\NormalTok{(df.test}\SpecialCharTok{$}\NormalTok{RS, pred\_RF)}
\end{Highlighting}
\end{Shaded}

\begin{verbatim}
## Warning in roc.default(df.test$RS, pred_RF): 'response' has more than two
## levels. Consider setting 'levels' explicitly or using 'multiclass.roc' instead
\end{verbatim}

\begin{verbatim}
## Setting levels: control = admis, case = exclusion
\end{verbatim}

\begin{verbatim}
## Setting direction: controls < cases
\end{verbatim}

\begin{Shaded}
\begin{Highlighting}[]
\NormalTok{ROC\_RF}\SpecialCharTok{$}\NormalTok{auc}
\end{Highlighting}
\end{Shaded}

\begin{verbatim}
## Area under the curve: 0.8535
\end{verbatim}

\begin{Shaded}
\begin{Highlighting}[]
\DocumentationTok{\#\# Accuracy}
\NormalTok{accuracy\_RF }\OtherTok{=} \FunctionTok{mean}\NormalTok{(df.test}\SpecialCharTok{$}\NormalTok{RS}\SpecialCharTok{==}\FunctionTok{predict}\NormalTok{(res\_RF,}\AttributeTok{newdata=}\NormalTok{df.test,}\AttributeTok{type=}\StringTok{"class"}\NormalTok{))}
\FunctionTok{print}\NormalTok{(}\StringTok{"accuracy RF = "}\NormalTok{)}
\end{Highlighting}
\end{Shaded}

\begin{verbatim}
## [1] "accuracy RF = "
\end{verbatim}

\begin{Shaded}
\begin{Highlighting}[]
\FunctionTok{print}\NormalTok{(accuracy\_RF)}
\end{Highlighting}
\end{Shaded}

\begin{verbatim}
## [1] 0.7177033
\end{verbatim}

\hypertarget{f-cart}{%
\subsection{f) CART}\label{f-cart}}

\begin{Shaded}
\begin{Highlighting}[]
\NormalTok{arbre }\OtherTok{=} \FunctionTok{rpart}\NormalTok{(df.train}\SpecialCharTok{$}\NormalTok{RS}\SpecialCharTok{\textasciitilde{}}\NormalTok{.,df.train,}\AttributeTok{control=}\FunctionTok{rpart.control}\NormalTok{(}\AttributeTok{minsplit=}\DecValTok{5}\NormalTok{,}\AttributeTok{cp=}\FloatTok{0.025}\NormalTok{))}
\NormalTok{cp.opt }\OtherTok{=}\NormalTok{ arbre}\SpecialCharTok{$}\NormalTok{cptable[}\FunctionTok{which.min}\NormalTok{(arbre}\SpecialCharTok{$}\NormalTok{cptable[, }\StringTok{"xerror"}\NormalTok{]), }\StringTok{"CP"}\NormalTok{]}
\NormalTok{res\_cart }\OtherTok{=} \FunctionTok{prune}\NormalTok{(arbre,}\AttributeTok{cp=}\NormalTok{cp.opt)}
\FunctionTok{rpart.plot}\NormalTok{(res\_cart)}
\end{Highlighting}
\end{Shaded}

\includegraphics{projet_files/figure-latex/unnamed-chunk-32-1.pdf}

\begin{Shaded}
\begin{Highlighting}[]
\DocumentationTok{\#\# prédiction :}
\NormalTok{pred\_cart }\OtherTok{\textless{}{-}} \FunctionTok{predict}\NormalTok{(res\_cart,}\AttributeTok{newdata=}\NormalTok{df.test)[,}\DecValTok{2}\NormalTok{] }

\DocumentationTok{\#\# Table confusion et accuracy :}
\FunctionTok{table}\NormalTok{(df.test}\SpecialCharTok{$}\NormalTok{RS, }\FunctionTok{predict}\NormalTok{(res\_cart,}\AttributeTok{newdata=}\NormalTok{df.test,}\AttributeTok{type=}\StringTok{"class"}\NormalTok{))}
\end{Highlighting}
\end{Shaded}

\begin{verbatim}
##               
##                admis exclusion redoublement
##   admis          133        10            0
##   exclusion       13        20            0
##   redoublement    24         9            0
\end{verbatim}

\begin{Shaded}
\begin{Highlighting}[]
\DocumentationTok{\#\# aire sous courbe ROC}
\NormalTok{pred\_cart }\OtherTok{=} \FunctionTok{predict}\NormalTok{(res\_cart, df.test, }\AttributeTok{type=}\StringTok{"prob"}\NormalTok{)[,}\DecValTok{2}\NormalTok{] }
\NormalTok{ROC\_cart }\OtherTok{\textless{}{-}} \FunctionTok{roc}\NormalTok{(df.test}\SpecialCharTok{$}\NormalTok{RS, pred\_cart)}
\end{Highlighting}
\end{Shaded}

\begin{verbatim}
## Warning in roc.default(df.test$RS, pred_cart): 'response' has more than two
## levels. Consider setting 'levels' explicitly or using 'multiclass.roc' instead
\end{verbatim}

\begin{verbatim}
## Setting levels: control = admis, case = exclusion
\end{verbatim}

\begin{verbatim}
## Setting direction: controls < cases
\end{verbatim}

\begin{Shaded}
\begin{Highlighting}[]
\NormalTok{ROC\_cart}\SpecialCharTok{$}\NormalTok{auc}
\end{Highlighting}
\end{Shaded}

\begin{verbatim}
## Area under the curve: 0.7681
\end{verbatim}

\begin{Shaded}
\begin{Highlighting}[]
\DocumentationTok{\#\# Accuracy}
\NormalTok{accuracy\_cart }\OtherTok{=} \FunctionTok{mean}\NormalTok{(df.test}\SpecialCharTok{$}\NormalTok{RS}\SpecialCharTok{==}\FunctionTok{predict}\NormalTok{(res\_cart,}\AttributeTok{newdata=}\NormalTok{df.test,}\AttributeTok{type=}\StringTok{"class"}\NormalTok{))}
\FunctionTok{print}\NormalTok{(}\StringTok{"accuracy cart = "}\NormalTok{)}
\end{Highlighting}
\end{Shaded}

\begin{verbatim}
## [1] "accuracy cart = "
\end{verbatim}

\begin{Shaded}
\begin{Highlighting}[]
\FunctionTok{print}\NormalTok{(accuracy\_cart)}
\end{Highlighting}
\end{Shaded}

\begin{verbatim}
## [1] 0.7320574
\end{verbatim}

\begin{enumerate}
\def\labelenumi{\alph{enumi})}
\setcounter{enumi}{7}
\tightlist
\item
  Adaboost
\end{enumerate}

\begin{Shaded}
\begin{Highlighting}[]
\CommentTok{\# fit.adaboost=gbm(as.numeric(RS){-}1 \textasciitilde{}., df.train, distribution = "adaboost")}
\CommentTok{\# fit.adaboost}
\CommentTok{\# }
\CommentTok{\# }\AlertTok{\#\#\#}\CommentTok{ Calibrer B=n.tree par cross{-}validation : }
\CommentTok{\# fit.adaboost=gbm(as.numeric(RS){-}1 \textasciitilde{}., df.train, distribution = "adaboost",cv.folds = 5, shrinkage = 0.01, n.trees=3000)}
\CommentTok{\# gbm.perf(fit.adaboost)}
\CommentTok{\# B.opt = gbm.perf(fit.adaboost, method="cv")}
\CommentTok{\# }
\CommentTok{\# \#\# prédiction : }
\CommentTok{\# pred\_adaboost = predict(fit.adaboost, newdata=df.test, type = "response", n.trees = B.opt)}
\CommentTok{\# class = 1*(pred\_adaboost\textgreater{}1/2)}
\CommentTok{\# }
\CommentTok{\# \#\# Table confusion et accuracy :}
\CommentTok{\# table(df.test$RS, class)}
\CommentTok{\# }
\CommentTok{\# \#\# Accuracy}
\CommentTok{\# accuracy\_adaboost = mean(as.numeric(df.test$RS){-}1==class)}
\CommentTok{\# print("accuracy adaboost = ")}
\CommentTok{\# print(accuracy\_adaboost)}
\CommentTok{\# }
\CommentTok{\# \#\# aire sous courbe ROC}
\CommentTok{\# ROC\_adaboost \textless{}{-} roc(df.test$RS, pred\_adaboost)}
\CommentTok{\# ROC\_adaboost$auc}
\end{Highlighting}
\end{Shaded}

\hypertarget{i-regression-logistique}{%
\subsection{i) Regression Logistique}\label{i-regression-logistique}}

\begin{Shaded}
\begin{Highlighting}[]
\DocumentationTok{\#\#\# Modèle}
\NormalTok{logit.train }\OtherTok{\textless{}{-}} \FunctionTok{glm}\NormalTok{(RS }\SpecialCharTok{\textasciitilde{}}\NormalTok{ ., }\AttributeTok{family =}\NormalTok{ binomial , }\AttributeTok{data=}\NormalTok{df.train)}

\DocumentationTok{\#\# prédiction :}
\NormalTok{pred\_logit }\OtherTok{\textless{}{-}} \FunctionTok{predict}\NormalTok{(logit.train,}\AttributeTok{newdata=}\NormalTok{df.test)}
\NormalTok{class }\OtherTok{=} \DecValTok{1}\SpecialCharTok{*}\NormalTok{(pred\_logit}\SpecialCharTok{\textgreater{}}\DecValTok{1}\SpecialCharTok{/}\DecValTok{2}\NormalTok{)}

\DocumentationTok{\#\# Table confusion et accuracy :}
\FunctionTok{table}\NormalTok{(df.test}\SpecialCharTok{$}\NormalTok{RS, class)}
\end{Highlighting}
\end{Shaded}

\begin{verbatim}
##               class
##                  0   1
##   admis        137   6
##   exclusion     19  14
##   redoublement  23  10
\end{verbatim}

\begin{Shaded}
\begin{Highlighting}[]
\DocumentationTok{\#\# aire sous courbe ROC}
\NormalTok{ROC\_logit }\OtherTok{\textless{}{-}} \FunctionTok{roc}\NormalTok{(df.test}\SpecialCharTok{$}\NormalTok{RS, pred\_logit)}
\end{Highlighting}
\end{Shaded}

\begin{verbatim}
## Warning in roc.default(df.test$RS, pred_logit): 'response' has more than two
## levels. Consider setting 'levels' explicitly or using 'multiclass.roc' instead
\end{verbatim}

\begin{verbatim}
## Setting levels: control = admis, case = exclusion
\end{verbatim}

\begin{verbatim}
## Setting direction: controls < cases
\end{verbatim}

\begin{Shaded}
\begin{Highlighting}[]
\DocumentationTok{\#\# Accuracy}
\NormalTok{accuracy\_logit }\OtherTok{=} \FunctionTok{mean}\NormalTok{(}\FunctionTok{as.numeric}\NormalTok{(df.test}\SpecialCharTok{$}\NormalTok{RS)}\SpecialCharTok{{-}}\DecValTok{1}\SpecialCharTok{==}\NormalTok{class)}
\FunctionTok{print}\NormalTok{(}\StringTok{"accuracy regression logistique = "}\NormalTok{)}
\end{Highlighting}
\end{Shaded}

\begin{verbatim}
## [1] "accuracy regression logistique = "
\end{verbatim}

\begin{Shaded}
\begin{Highlighting}[]
\FunctionTok{print}\NormalTok{(accuracy\_logit)}
\end{Highlighting}
\end{Shaded}

\begin{verbatim}
## [1] 0.722488
\end{verbatim}

\begin{Shaded}
\begin{Highlighting}[]
\NormalTok{ROC\_logit}\SpecialCharTok{$}\NormalTok{auc}
\end{Highlighting}
\end{Shaded}

\begin{verbatim}
## Area under the curve: 0.8339
\end{verbatim}

\begin{Shaded}
\begin{Highlighting}[]
\CommentTok{\# \# régression logistique Lasso}
\CommentTok{\# library(glmnet)}
\CommentTok{\# res\_Lasso \textless{}{-} glmnet(as.matrix(df.train[,{-}1]),df.train$RS,family=\textquotesingle{}binomial\textquotesingle{}) }
\CommentTok{\# plot(res\_Lasso, label = TRUE)  \# en abscisse : norme des coefficients}
\CommentTok{\# plot(res\_Lasso, xvar = "lambda", label = TRUE) \# en abscisse : log(lambda)}
\CommentTok{\# \# sum(coef(res\_Lasso, s=exp())!=0)}
\CommentTok{\# }
\CommentTok{\# cvLasso \textless{}{-} cv.glmnet(as.matrix(df.train[,{-}1]),df.train$RS,family="binomial", type.measure = "class") }
\CommentTok{\# plot(cvLasso)}
\CommentTok{\# cvLasso$lambda.min}
\CommentTok{\# coef(res\_Lasso, s=cvLasso$lambda.min)}
\CommentTok{\# }
\CommentTok{\# \#prédiction}
\CommentTok{\# class\_logit\_lasso=predict(cvLasso, newx = as.matrix(df.test[,{-}1]), s = \textquotesingle{}lambda.min\textquotesingle{}, type = "class")}
\CommentTok{\# }
\CommentTok{\# \#Table de confusion et accuracy}
\CommentTok{\# table(df.test$RS, class\_logit\_lasso)}
\CommentTok{\# pred\_logit\_lasso=predict(cvLasso, newx = as.matrix(df.test[,{-}1]), s = \textquotesingle{}lambda.min\textquotesingle{}, type = "response")}
\CommentTok{\# }
\CommentTok{\# accuracy\_logit\_lasso = mean(df.test$RS==class\_logit\_lasso)}
\CommentTok{\# print("accuracy regression logistique lasso= ")}
\CommentTok{\# print(accuracy\_logit\_lasso)}
\CommentTok{\# }
\CommentTok{\# \#pred\_logit\_lasso}
\CommentTok{\# ROC\_logit\_lasso = roc( df.test$RS, pred\_logit\_lasso)}
\CommentTok{\# ROC\_logit\_lasso$auc}
\end{Highlighting}
\end{Shaded}

\hypertarget{comparaison}{%
\subsection{Comparaison}\label{comparaison}}

\begin{Shaded}
\begin{Highlighting}[]
\NormalTok{result}\OtherTok{=}\FunctionTok{matrix}\NormalTok{(}\ConstantTok{NA}\NormalTok{, }\AttributeTok{ncol=}\DecValTok{5}\NormalTok{, }\AttributeTok{nrow=}\DecValTok{2}\NormalTok{)}
\FunctionTok{rownames}\NormalTok{(result)}\OtherTok{=}\FunctionTok{c}\NormalTok{(}\StringTok{\textquotesingle{}accuracy\textquotesingle{}}\NormalTok{, }\StringTok{\textquotesingle{}AUC\textquotesingle{}}\NormalTok{)}
\FunctionTok{colnames}\NormalTok{(result)}\OtherTok{=}\FunctionTok{c}\NormalTok{(}\StringTok{\textquotesingle{}lda\textquotesingle{}}\NormalTok{, }\StringTok{\textquotesingle{}qda\textquotesingle{}}\NormalTok{, }\StringTok{\textquotesingle{}cart\textquotesingle{}}\NormalTok{, }\StringTok{\textquotesingle{}RF\textquotesingle{}}\NormalTok{,}\StringTok{"logit"}\NormalTok{)}
\NormalTok{result[}\DecValTok{1}\NormalTok{,]}\OtherTok{=} \FunctionTok{c}\NormalTok{(accuracy\_lda, accuracy\_qda, accuracy\_cart, accuracy\_RF,accuracy\_logit)}
\NormalTok{result[}\DecValTok{2}\NormalTok{,]}\OtherTok{=}\FunctionTok{c}\NormalTok{(ROC\_lda}\SpecialCharTok{$}\NormalTok{auc, ROC\_qda}\SpecialCharTok{$}\NormalTok{auc, ROC\_cart}\SpecialCharTok{$}\NormalTok{auc, ROC\_RF}\SpecialCharTok{$}\NormalTok{auc,ROC\_logit}\SpecialCharTok{$}\NormalTok{auc)}
\NormalTok{result}
\end{Highlighting}
\end{Shaded}

\begin{verbatim}
##                lda       qda      cart        RF     logit
## accuracy 0.7511962 0.7272727 0.7320574 0.7177033 0.7224880
## AUC      0.8226319 0.6944268 0.7680653 0.8534647 0.8338631
\end{verbatim}

\begin{Shaded}
\begin{Highlighting}[]
\FunctionTok{apply}\NormalTok{(result,}\DecValTok{1}\NormalTok{, which.max )}
\end{Highlighting}
\end{Shaded}

\begin{verbatim}
## accuracy      AUC 
##        1        4
\end{verbatim}

\begin{Shaded}
\begin{Highlighting}[]
\FunctionTok{plot}\NormalTok{(ROC\_lda, }\AttributeTok{xlim=}\FunctionTok{c}\NormalTok{(}\DecValTok{1}\NormalTok{,}\DecValTok{0}\NormalTok{))}
\FunctionTok{plot}\NormalTok{(ROC\_qda, }\AttributeTok{add=}\ConstantTok{TRUE}\NormalTok{, }\AttributeTok{col=}\DecValTok{2}\NormalTok{)}
\FunctionTok{plot}\NormalTok{(ROC\_cart, }\AttributeTok{add=}\ConstantTok{TRUE}\NormalTok{, }\AttributeTok{col=}\DecValTok{3}\NormalTok{)}
\FunctionTok{plot}\NormalTok{(ROC\_RF, }\AttributeTok{add=}\ConstantTok{TRUE}\NormalTok{, }\AttributeTok{col=}\DecValTok{4}\NormalTok{)}
\FunctionTok{plot}\NormalTok{(ROC\_logit, }\AttributeTok{add=}\ConstantTok{TRUE}\NormalTok{, }\AttributeTok{col=}\DecValTok{6}\NormalTok{)}
\FunctionTok{legend}\NormalTok{(}\StringTok{\textquotesingle{}bottom\textquotesingle{}}\NormalTok{, }\AttributeTok{col=}\DecValTok{1}\SpecialCharTok{:}\DecValTok{5}\NormalTok{, }\FunctionTok{paste}\NormalTok{(}\FunctionTok{c}\NormalTok{(}\StringTok{\textquotesingle{}lda\textquotesingle{}}\NormalTok{, }\StringTok{\textquotesingle{}qda\textquotesingle{}}\NormalTok{, }\StringTok{\textquotesingle{}cart\textquotesingle{}}\NormalTok{, }\StringTok{\textquotesingle{}RF\textquotesingle{}}\NormalTok{,}\StringTok{"logit"}\NormalTok{)),  }\AttributeTok{lwd=}\DecValTok{1}\NormalTok{)}
\end{Highlighting}
\end{Shaded}

\includegraphics{projet_files/figure-latex/unnamed-chunk-36-1.pdf}
\#\#ACP

\begin{Shaded}
\begin{Highlighting}[]
\NormalTok{data\_quanti}\OtherTok{=}\NormalTok{df[,}\FunctionTok{c}\NormalTok{(}\DecValTok{3}\NormalTok{,}\DecValTok{13}\NormalTok{,}\DecValTok{14}\NormalTok{,}\DecValTok{15}\NormalTok{,}\DecValTok{31}\NormalTok{,}\DecValTok{32}\NormalTok{,}\DecValTok{33}\NormalTok{,}\DecValTok{34}\NormalTok{)]}
\FunctionTok{head}\NormalTok{(data\_quanti)}
\end{Highlighting}
\end{Shaded}

\begin{verbatim}
##   age traveltime studytime failures G1 G2 G3       Moy
## 1  18          2         2        0  5  6  6  5.666667
## 2  17          1         2        0  5  5  6  5.333333
## 3  15          1         2        3  7  8 10  8.333333
## 4  15          1         3        0 15 14 15 14.666667
## 5  16          1         2        0  6 10 10  8.666667
## 6  16          1         2        0 15 15 15 15.000000
\end{verbatim}

On va par la suite transformer lorsque cela est possible certaines
variables qualitatives en variables quantitatives afin de pouvoir
réaliser une ACP dessus. Pour les variables studytime et traveltime, des
intervalles nous sont données, on prend donc pour chaque niveau le
millieu de l'intervalle. Pour les valeurs extrèmes, 1 et 4, on choisit
arbitrairement une borne supèrieure ou inférieure (15H00 pour studytime
et 3h00 pour traveltime pour ce qu'il s'agit des bornes supérieures et
0h00 pour les deux bornes inférieures)

\begin{Shaded}
\begin{Highlighting}[]
\CommentTok{\#on convertit studytime et travel time en variables quantitatives (on prend le millieux des segments)}
\ControlFlowTok{for}\NormalTok{ (i }\ControlFlowTok{in} \DecValTok{1}\SpecialCharTok{:}\FunctionTok{nrow}\NormalTok{(data\_quanti))\{}
  \ControlFlowTok{if}\NormalTok{ (data\_quanti}\SpecialCharTok{$}\NormalTok{studytime[i]}\SpecialCharTok{==}\DecValTok{2}\NormalTok{)\{}
\NormalTok{    data\_quanti}\SpecialCharTok{$}\NormalTok{studytime[i]}\OtherTok{=}\DecValTok{210}
\NormalTok{  \}}
   \ControlFlowTok{if}\NormalTok{ (data\_quanti}\SpecialCharTok{$}\NormalTok{studytime[i]}\SpecialCharTok{==}\DecValTok{1}\NormalTok{)\{}
\NormalTok{    data\_quanti}\SpecialCharTok{$}\NormalTok{studytime[i]}\OtherTok{=}\DecValTok{120}
\NormalTok{   \}}
   \ControlFlowTok{if}\NormalTok{ (data\_quanti}\SpecialCharTok{$}\NormalTok{studytime[i]}\SpecialCharTok{==}\DecValTok{3}\NormalTok{)\{}
\NormalTok{    data\_quanti}\SpecialCharTok{$}\NormalTok{studytime[i]}\OtherTok{=}\DecValTok{450}
\NormalTok{   \}}
   \ControlFlowTok{if}\NormalTok{ (data\_quanti}\SpecialCharTok{$}\NormalTok{studytime[i]}\SpecialCharTok{==}\DecValTok{4}\NormalTok{)\{}
\NormalTok{    data\_quanti}\SpecialCharTok{$}\NormalTok{studytime[i]}\OtherTok{=}\DecValTok{750}
\NormalTok{   \}}
  \ControlFlowTok{if}\NormalTok{(data\_quanti}\SpecialCharTok{$}\NormalTok{traveltime[i]}\SpecialCharTok{==}\DecValTok{1}\NormalTok{)\{}
\NormalTok{    data\_quanti}\SpecialCharTok{$}\NormalTok{traveltime[i]}\OtherTok{=}\FloatTok{7.5}
\NormalTok{  \}}
   \ControlFlowTok{if}\NormalTok{(data\_quanti}\SpecialCharTok{$}\NormalTok{traveltime[i]}\SpecialCharTok{==}\DecValTok{2}\NormalTok{)\{}
\NormalTok{    data\_quanti}\SpecialCharTok{$}\NormalTok{traveltime[i]}\OtherTok{=}\FloatTok{22.5}
\NormalTok{   \}}
   \ControlFlowTok{if}\NormalTok{(data\_quanti}\SpecialCharTok{$}\NormalTok{traveltime[i]}\SpecialCharTok{==}\DecValTok{3}\NormalTok{)\{}
\NormalTok{    data\_quanti}\SpecialCharTok{$}\NormalTok{traveltime[i]}\OtherTok{=}\DecValTok{45}
\NormalTok{   \}}
   \ControlFlowTok{if}\NormalTok{(data\_quanti}\SpecialCharTok{$}\NormalTok{traveltime[i]}\SpecialCharTok{==}\DecValTok{4}\NormalTok{)\{}
\NormalTok{    data\_quanti}\SpecialCharTok{$}\NormalTok{traveltime[i]}\OtherTok{=}\DecValTok{120}
\NormalTok{  \}}
  
  
\NormalTok{\}}
\FunctionTok{head}\NormalTok{(data\_quanti)}
\end{Highlighting}
\end{Shaded}

\begin{verbatim}
##   age traveltime studytime failures G1 G2 G3       Moy
## 1  18       22.5       210        0  5  6  6  5.666667
## 2  17        7.5       210        0  5  5  6  5.333333
## 3  15        7.5       210        3  7  8 10  8.333333
## 4  15        7.5       450        0 15 14 15 14.666667
## 5  16        7.5       210        0  6 10 10  8.666667
## 6  16        7.5       210        0 15 15 15 15.000000
\end{verbatim}

\begin{Shaded}
\begin{Highlighting}[]
\NormalTok{res}\OtherTok{=}\FunctionTok{PCA}\NormalTok{(data\_quanti)}
\end{Highlighting}
\end{Shaded}

\includegraphics{projet_files/figure-latex/unnamed-chunk-39-1.pdf}
\includegraphics{projet_files/figure-latex/unnamed-chunk-39-2.pdf} On
voit que les variables study time et travel time sont mal projetées, on
ne peut donc pas les interprétes. De manière logique on retrouve que les
élèves ayant une bonne moyenne ont eu un bonne note à chaque semestres
Vers la gauche se trouvent les paramètres ayant une influence négatives
que la moyenne comme les echecs et plus curieusement l'âge (peut être
sagit il des personnes ayant redoubler).

\begin{Shaded}
\begin{Highlighting}[]
\NormalTok{res}\SpecialCharTok{$}\NormalTok{eig}
\end{Highlighting}
\end{Shaded}

\begin{verbatim}
##          eigenvalue percentage of variance cumulative percentage of variance
## comp 1 4.018158e+00           5.022697e+01                          50.22697
## comp 2 1.111100e+00           1.388875e+01                          64.11572
## comp 3 9.810219e-01           1.226277e+01                          76.37850
## comp 4 9.605297e-01           1.200662e+01                          88.38512
## comp 5 6.518265e-01           8.147831e+00                          96.53295
## comp 6 1.970800e-01           2.463500e+00                          98.99645
## comp 7 8.028406e-02           1.003551e+00                         100.00000
## comp 8 9.728316e-31           1.216040e-29                         100.00000
\end{verbatim}

On ne garde que deux dimensions ici, d'ou l'analyse ci dessus

\begin{Shaded}
\begin{Highlighting}[]
\FunctionTok{plot}\NormalTok{(res, }\AttributeTok{select=}\StringTok{"cos2 0.8"}\NormalTok{, }\AttributeTok{habillage=}\DecValTok{3}\NormalTok{, }\AttributeTok{cex=}\FloatTok{0.9}\NormalTok{,}\AttributeTok{choix=}\StringTok{"ind"}\NormalTok{)}\CommentTok{\#on visualise le temps de travail}
\end{Highlighting}
\end{Shaded}

\includegraphics{projet_files/figure-latex/unnamed-chunk-41-1.pdf} On
retrouve bien que les personnes ayant le plus travaillé se situé di coté
des bonnes notes.

\begin{Shaded}
\begin{Highlighting}[]
\FunctionTok{plot}\NormalTok{(res, }\AttributeTok{select=}\StringTok{"cos2 0.8"}\NormalTok{, }\AttributeTok{habillage=}\DecValTok{2}\NormalTok{, }\AttributeTok{cex=}\FloatTok{0.9}\NormalTok{,}\AttributeTok{choix=}\StringTok{"ind"}\NormalTok{)}\CommentTok{\#visualisation du temps de trajet}
\end{Highlighting}
\end{Shaded}

\includegraphics{projet_files/figure-latex/unnamed-chunk-42-1.pdf} De la
même manière on voit que le temps de trajet a un influence négative sur
la réussite.

\begin{Shaded}
\begin{Highlighting}[]
\FunctionTok{plot}\NormalTok{(res, }\AttributeTok{select=}\StringTok{"cos2 0.8"}\NormalTok{, }\AttributeTok{habillage=}\DecValTok{4}\NormalTok{, }\AttributeTok{cex=}\FloatTok{0.9}\NormalTok{,}\AttributeTok{choix=}\StringTok{"ind"}\NormalTok{)}\CommentTok{\#visualisation des échecs}
\end{Highlighting}
\end{Shaded}

\includegraphics{projet_files/figure-latex/unnamed-chunk-43-1.pdf} On
voit aussi que les élèves qui ont les meilleurs résulats sont ceux qui
ont le moins d'échecs. Par ailleurs l'acp ici ne semble pas très
pertinente car la plupart des variables du jeu de données sont
quantitatives, nous avons donc été obligés de les rendre (lorsque cela a
un sens) qualitatives. Néeanmoins on voit par exemple que pour ces
variables transformées, leur projection est très mauvaise et ne peuvent
donc pas être interprétes à l'aide de l'ACP (comme studytime et
traveltime). Egalement peut être qu'il y a une meilleure de les rendres
qualitatives. C'est pour quoi l'on va réaliser par la suite un anova 2
sur les variables quatitatives studytime et traveltime afin de pouvoir
expliqués la variable Moy avec. \#\#Anova 2 sur les variables studytime
et traveltime

\begin{Shaded}
\begin{Highlighting}[]
\CommentTok{\#création de la data frame correspondante}
\NormalTok{data\_anova}\OtherTok{=}\NormalTok{df[,}\FunctionTok{c}\NormalTok{(}\DecValTok{13}\NormalTok{,}\DecValTok{14}\NormalTok{,}\DecValTok{34}\NormalTok{)]}
\NormalTok{data\_anova}\SpecialCharTok{$}\NormalTok{traveltime}\OtherTok{=}\FunctionTok{factor}\NormalTok{(data\_anova}\SpecialCharTok{$}\NormalTok{traveltime)}
\NormalTok{data\_anova}\SpecialCharTok{$}\NormalTok{studytime}\OtherTok{=}\FunctionTok{factor}\NormalTok{(data\_anova}\SpecialCharTok{$}\NormalTok{studytime)}
\FunctionTok{attach}\NormalTok{(data\_anova)}
\end{Highlighting}
\end{Shaded}

\begin{verbatim}
## Les objets suivants sont masqués depuis data_quanti:
## 
##     studytime, traveltime
\end{verbatim}

\begin{Shaded}
\begin{Highlighting}[]
\FunctionTok{head}\NormalTok{(data\_anova)}
\end{Highlighting}
\end{Shaded}

\begin{verbatim}
##   traveltime studytime       Moy
## 1          2         2  5.666667
## 2          1         2  5.333333
## 3          1         2  8.333333
## 4          1         3 14.666667
## 5          1         2  8.666667
## 6          1         2 15.000000
\end{verbatim}

\begin{Shaded}
\begin{Highlighting}[]
\FunctionTok{table}\NormalTok{(data\_anova}\SpecialCharTok{$}\NormalTok{traveltime,data\_anova}\SpecialCharTok{$}\NormalTok{studytime)}
\end{Highlighting}
\end{Shaded}

\begin{verbatim}
##    
##       1   2   3   4
##   1 165 314 108  36
##   2 110 143  46  21
##   3  34  37   4   2
##   4   8   9   4   3
\end{verbatim}

Le plan est trop désiquilibré pour faire un anova \#\#Anova 2 sur les
variables romantic et Walc

\begin{Shaded}
\begin{Highlighting}[]
\CommentTok{\#création de la data frame correspondante}
\FunctionTok{head}\NormalTok{(df)}
\end{Highlighting}
\end{Shaded}

\begin{verbatim}
##   school sex age address famsize Pstatus Medu Fedu     Mjob     Fjob     reason
## 1     GP   F  18       U     GT3       A    4    4  at_home  teacher     course
## 2     GP   F  17       U     GT3       T    1    1  at_home    other     course
## 3     GP   F  15       U     LE3       T    1    1  at_home    other      other
## 4     GP   F  15       U     GT3       T    4    2   health services       home
## 5     GP   F  16       U     GT3       T    3    3    other    other       home
## 6     GP   M  16       U     LE3       T    4    3 services    other reputation
##   guardian traveltime studytime failures schoolsup famsup paid activities
## 1   mother          2         2        0       yes     no   no         no
## 2   father          1         2        0        no    yes   no         no
## 3   mother          1         2        3       yes     no  yes         no
## 4   mother          1         3        0        no    yes  yes        yes
## 5   father          1         2        0        no    yes  yes         no
## 6   mother          1         2        0        no    yes  yes        yes
##   nursery higher internet romantic famrel freetime goout Dalc Walc health
## 1     yes    yes       no       no      4        3     4    1    1      3
## 2      no    yes      yes       no      5        3     3    1    1      3
## 3     yes    yes      yes       no      4        3     2    2    3      3
## 4     yes    yes      yes      yes      3        2     2    1    1      5
## 5     yes    yes       no       no      4        3     2    1    2      5
## 6     yes    yes      yes       no      5        4     2    1    2      5
##   absences G1 G2 G3       Moy           RS
## 1        6  5  6  6  5.666667    exclusion
## 2        4  5  5  6  5.333333    exclusion
## 3       10  7  8 10  8.333333    exclusion
## 4        2 15 14 15 14.666667        admis
## 5        4  6 10 10  8.666667 redoublement
## 6       10 15 15 15 15.000000        admis
\end{verbatim}

\begin{Shaded}
\begin{Highlighting}[]
\NormalTok{data\_anova}\OtherTok{=}\NormalTok{df[,}\FunctionTok{c}\NormalTok{(}\DecValTok{28}\NormalTok{,}\DecValTok{23}\NormalTok{,}\DecValTok{34}\NormalTok{)]}
\NormalTok{data\_anova}\SpecialCharTok{$}\NormalTok{Walc}\OtherTok{=}\FunctionTok{factor}\NormalTok{(data\_anova}\SpecialCharTok{$}\NormalTok{Walc)}
\FunctionTok{head}\NormalTok{(data\_anova)}
\end{Highlighting}
\end{Shaded}

\begin{verbatim}
##   Walc romantic       Moy
## 1    1       no  5.666667
## 2    1       no  5.333333
## 3    3       no  8.333333
## 4    1      yes 14.666667
## 5    2       no  8.666667
## 6    2       no 15.000000
\end{verbatim}

\begin{Shaded}
\begin{Highlighting}[]
\FunctionTok{table}\NormalTok{(data\_anova}\SpecialCharTok{$}\NormalTok{Walc,data\_anova}\SpecialCharTok{$}\NormalTok{romantic)}
\end{Highlighting}
\end{Shaded}

\begin{verbatim}
##    
##      no yes
##   1 253 145
##   2 151  84
##   3 127  73
##   4  98  40
##   5  44  29
\end{verbatim}

Le modèle est complet et n'est pas trop désiquilibré.

\begin{Shaded}
\begin{Highlighting}[]
\NormalTok{res}\OtherTok{=}\FunctionTok{lm}\NormalTok{(Moy}\SpecialCharTok{\textasciitilde{}}\NormalTok{romantic}\SpecialCharTok{*}\NormalTok{Walc,data\_anova)}
\FunctionTok{par}\NormalTok{(}\AttributeTok{mfrow=}\FunctionTok{c}\NormalTok{(}\DecValTok{2}\NormalTok{,}\DecValTok{2}\NormalTok{))}
\FunctionTok{plot}\NormalTok{(res)}
\end{Highlighting}
\end{Shaded}

\includegraphics{projet_files/figure-latex/unnamed-chunk-48-1.pdf}

\begin{Shaded}
\begin{Highlighting}[]
\FunctionTok{shapiro.test}\NormalTok{(res}\SpecialCharTok{$}\NormalTok{residuals)}
\end{Highlighting}
\end{Shaded}

\begin{verbatim}
## 
##  Shapiro-Wilk normality test
## 
## data:  res$residuals
## W = 0.98902, p-value = 4.858e-07
\end{verbatim}

Les données ne sont pas du tout gaussiennes.

\begin{Shaded}
\begin{Highlighting}[]
\FunctionTok{head}\NormalTok{(df)}
\end{Highlighting}
\end{Shaded}

\begin{verbatim}
##   school sex age address famsize Pstatus Medu Fedu     Mjob     Fjob     reason
## 1     GP   F  18       U     GT3       A    4    4  at_home  teacher     course
## 2     GP   F  17       U     GT3       T    1    1  at_home    other     course
## 3     GP   F  15       U     LE3       T    1    1  at_home    other      other
## 4     GP   F  15       U     GT3       T    4    2   health services       home
## 5     GP   F  16       U     GT3       T    3    3    other    other       home
## 6     GP   M  16       U     LE3       T    4    3 services    other reputation
##   guardian traveltime studytime failures schoolsup famsup paid activities
## 1   mother          2         2        0       yes     no   no         no
## 2   father          1         2        0        no    yes   no         no
## 3   mother          1         2        3       yes     no  yes         no
## 4   mother          1         3        0        no    yes  yes        yes
## 5   father          1         2        0        no    yes  yes         no
## 6   mother          1         2        0        no    yes  yes        yes
##   nursery higher internet romantic famrel freetime goout Dalc Walc health
## 1     yes    yes       no       no      4        3     4    1    1      3
## 2      no    yes      yes       no      5        3     3    1    1      3
## 3     yes    yes      yes       no      4        3     2    2    3      3
## 4     yes    yes      yes      yes      3        2     2    1    1      5
## 5     yes    yes       no       no      4        3     2    1    2      5
## 6     yes    yes      yes       no      5        4     2    1    2      5
##   absences G1 G2 G3       Moy           RS
## 1        6  5  6  6  5.666667    exclusion
## 2        4  5  5  6  5.333333    exclusion
## 3       10  7  8 10  8.333333    exclusion
## 4        2 15 14 15 14.666667        admis
## 5        4  6 10 10  8.666667 redoublement
## 6       10 15 15 15 15.000000        admis
\end{verbatim}

\begin{Shaded}
\begin{Highlighting}[]
\NormalTok{data\_anova}\OtherTok{=}\NormalTok{df[}\FunctionTok{c}\NormalTok{(}\DecValTok{34}\NormalTok{,}\DecValTok{18}\NormalTok{,}\DecValTok{22}\NormalTok{)]}
\FunctionTok{head}\NormalTok{(data\_anova)}
\end{Highlighting}
\end{Shaded}

\begin{verbatim}
##         Moy paid internet
## 1  5.666667   no       no
## 2  5.333333   no      yes
## 3  8.333333  yes      yes
## 4 14.666667  yes      yes
## 5  8.666667  yes       no
## 6 15.000000  yes      yes
\end{verbatim}

\begin{Shaded}
\begin{Highlighting}[]
\FunctionTok{table}\NormalTok{(data\_anova}\SpecialCharTok{$}\NormalTok{internet,data\_anova}\SpecialCharTok{$}\NormalTok{paid)}
\end{Highlighting}
\end{Shaded}

\begin{verbatim}
##      
##        no yes
##   no  191  26
##   yes 633 194
\end{verbatim}

Le plan est complet et quasiment équilibré

\begin{Shaded}
\begin{Highlighting}[]
\NormalTok{res}\OtherTok{=}\FunctionTok{lm}\NormalTok{(Moy}\SpecialCharTok{\textasciitilde{}}\NormalTok{internet}\SpecialCharTok{*}\NormalTok{paid,data\_anova)}
\FunctionTok{par}\NormalTok{(}\AttributeTok{mfrow=}\FunctionTok{c}\NormalTok{(}\DecValTok{2}\NormalTok{,}\DecValTok{2}\NormalTok{))}
\FunctionTok{plot}\NormalTok{(res)}
\end{Highlighting}
\end{Shaded}

\includegraphics{projet_files/figure-latex/unnamed-chunk-52-1.pdf}

\begin{Shaded}
\begin{Highlighting}[]
\FunctionTok{shapiro.test}\NormalTok{(res}\SpecialCharTok{$}\NormalTok{residuals)}
\end{Highlighting}
\end{Shaded}

\begin{verbatim}
## 
##  Shapiro-Wilk normality test
## 
## data:  res$residuals
## W = 0.99028, p-value = 2.162e-06
\end{verbatim}

On obtiens encore que les données ne sont pas gaussiennes

\#\#Modèle linéaire Gaussien: Régréssion mutliple

\begin{Shaded}
\begin{Highlighting}[]
\FunctionTok{library}\NormalTok{(car) }\CommentTok{\#pour utiliser VIF}
\end{Highlighting}
\end{Shaded}

\begin{verbatim}
## Le chargement a nécessité le package : carData
\end{verbatim}

\begin{verbatim}
## 
## Attachement du package : 'car'
\end{verbatim}

\begin{verbatim}
## L'objet suivant est masqué depuis 'package:dplyr':
## 
##     recode
\end{verbatim}

\begin{Shaded}
\begin{Highlighting}[]
\NormalTok{reg}\OtherTok{=}\FunctionTok{lm}\NormalTok{(Moy}\SpecialCharTok{\textasciitilde{}}\NormalTok{.,data\_quanti)}\CommentTok{\#regression multiple pour utiliser VIF }
\FunctionTok{vif}\NormalTok{(reg)}\CommentTok{\#test de colinéarité }
\end{Highlighting}
\end{Shaded}

\begin{verbatim}
## Warning in summary.lm(object, ...): essentially perfect fit: summary may be
## unreliable
\end{verbatim}

\begin{verbatim}
##        age traveltime  studytime   failures         G1         G2         G3 
##   1.087837   1.027492   1.048104   1.278165   3.943316   7.959427   6.069691
\end{verbatim}

Aucune valeur n'est plsu grande que 10, la matrice est donc de plein
rang. On va maintenant vérifier si les résidus sont iid, gaussiens
centrée et réduits

\begin{Shaded}
\begin{Highlighting}[]
\FunctionTok{par}\NormalTok{(}\AttributeTok{mfrow=}\FunctionTok{c}\NormalTok{(}\DecValTok{2}\NormalTok{,}\DecValTok{2}\NormalTok{))}
\FunctionTok{plot}\NormalTok{(reg)}
\end{Highlighting}
\end{Shaded}

\includegraphics{projet_files/figure-latex/unnamed-chunk-55-1.pdf} On
voit qu'il n'y a pas de forme de trompette sur le graphe des résidus don
l'hypoithèse d'homoscedasticité est vérifiée. Néanmoins il semble y
avoir plusieurs points avec des résidus trop grands.

\begin{Shaded}
\begin{Highlighting}[]
\FunctionTok{abs}\NormalTok{(}\FunctionTok{rstudent}\NormalTok{(reg))[}\FunctionTok{abs}\NormalTok{(}\FunctionTok{rstudent}\NormalTok{(reg))}\SpecialCharTok{\textgreater{}}\DecValTok{2}\NormalTok{]}
\end{Highlighting}
\end{Shaded}

\begin{verbatim}
##         1         2         4         6         8        48 
## 40.270194  2.823092  3.837576 12.191489  6.387979  2.082211
\end{verbatim}

En effet, on voit qu'il y en a huit. Il faudrais enlever le point le
plus éloigé. Néanmoins, on voit en regardant le qqplot nos variables
n'ont aucune chance d'être gaussiennes. En effet, avec la p-valeur du
test de shapiro qui est très petite devant 5\%, on rejette H0, les
données ne sont donc pas gausssiennes. Le modèle n'est donc pas adpaté.

\begin{Shaded}
\begin{Highlighting}[]
\FunctionTok{shapiro.test}\NormalTok{(reg}\SpecialCharTok{$}\NormalTok{residuals)}
\end{Highlighting}
\end{Shaded}

\begin{verbatim}
## 
##  Shapiro-Wilk normality test
## 
## data:  reg$residuals
## W = 0.35748, p-value < 2.2e-16
\end{verbatim}

\end{document}
